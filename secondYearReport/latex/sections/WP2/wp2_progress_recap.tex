%!TEX root = ../../secondYearReport.tex


 
\paragraph*{WP2: understanding and modelling human whole-body behaviours in physical interaction (JSI)}

After the second year of project, all planed objectives have been reached and one experimental modification has been implemented.

A thorough review was created with summary of the recent literature on human postural control and whole body motion in contact with environment. It includes relevant publications up to date and reviews the methods for evaluation of postural stability of bipedal systems beyond available reviews \cite{Mergner2007, Azevedo2007}. The review provides a solid bridge in methodologies and terminologies used by the project partners from multi-disciplinary backgrounds. Based on the overall objectives of the project and the specific objectives of WP2, experimental setup was created and procedures were defined. We obtained ethics committee approval for all project related human experiments (approved by National Medical Ethics Committee of Republic of Slovenia, reference number 112/06/13).

We performed an experimental study and examined functional role of supportive hand contact at different locations where balance of an individual was perturbed by translational perturbations of the support surface. We examined the effects of handle location, perturbation direction and perturbation intensity on the postural control and the forces generated in the handle. We found that an additional supportive hand contact significantly reduced the maximal displacement of the subject's centre of pressure (CoP) regardless of the position of the handle, direction of the perturbation and its intensity. This is in agreement with the accepted belief that an additional hand contact with support in general reduces the destabilizing effect of balance perturbation \cite{Maki1997, Bateni2005, Maki2006, Wing2011}. On the other hand, the position of the handle had no effects on the maximal CoP displacement. This supports the idea that maintaining postural stability is the task of the highest priority and that the central nervous system does whatever necessary to keep the body balanced \cite{Winter1995}. Our findings are in contrast with the recent findings of Sarraf et al. \cite{Sarraf2014}. We submitted a manuscript explaining the results of our study to Gait \& Posture Journal. The manuscript is under review after minor revision and is expected to be published by the end of 2014 \cite{Babic2014}.

Most of the human studies that examine postural control \cite{Horak1986, Henry1998, Dimitrova2004} (including our above mentioned study) utilize one-time support perturbations that unpredictably perturb the balance of an individual. During our experiments we noticed that human subjects reacted to all such perturbations regardless of how small or slow the perturbation was or what was the initial acceleration of the perturbation. Our conclusion was that these reactions are essentially protective reactions that do not necessarily have counterbalancing effects \cite{McIlroy1995, Corbeil2013}. Such reactions mask the real factors involved in human choice of contact utilization. We therefore altered the perturbation methods for our further experiments and designed continuous perturbations in a frequency band that corresponds with typical human motion during postural control \cite{Nawayseh2006}.


