%!TEX root = ../../secondYearReport.tex


\paragraph{Work package 4 progress}

\subparagraph{Improved Models from Real-Time Regression with Latent Contact Type Inference (T4.1)}

Within T4.1 IIT developed in the first and the second year of the 
project a theoretical framework for estimating whole-body
dynamics from distributed multimodal sensors \cite{nori2015}. The sensors considered
include joint encoders, gyroscopes, accelerometers and force/torque sensors. In the third year of the project, IIT investigated the integration of this estimation techniques with the classical identification techniques for inertial parameters \cite{traversaro2015parametersEM}. For the fourth year the improvement of the reliability of the sensors was considered. More specifically the six axis force/torque sensors where discovered to have a change in behaviour after they are mounted on the robot. To tackle this an in-situ calibration procedure was developed.

The procedure used to calibrate the sensors in-situ takes advantage of the knowledge of the model of the robot to generate the expected wrenches of the sensors during some arbitrary motions. The wrenches used as reference are estimated through the model and kinematic measurements. It then uses this information to train and validate new calibration matrices, taking into account the calibration matrix obtained with a classical Workbench calibration.  This procedure was validated  on the F/T sensors mounted on the iCub humanoid robot legs and published in Humanoids 2016 \cite{Andrade}.


\begin{figure}
        \centering
        \includegraphics[width=.45\textwidth]{images/all1valid2.png}
        \caption{3D force comparison among the calibration matrices trained on each dataset against the model estimated forces on the fastExtBal1 dataset. }
        \label{fig:all1valid}
\end{figure}



% bibliography 
%@inproceedings{nori2015,
%  title={Simultaneous state and dynamics estimation in articulated structures},
%  author={Nori, Francesco and Kuppuswamy, Naveen and Traversaro, Silvio},
%  booktitle={Intelligent Robots and Systems (IROS), 2015 IEEE/RSJ International Conference on},
%  pages={3380--3386},
%  year={2015},
%  organization={IEEE}
%}
%
%@article{traversaro2015parametersEM,
%  title={Dynamic parameters identification in articulated rigid bodies with redundant heterogeneous sensors},
%  author={Traversaro, Silvio and Venture, Gentiane and Nori, Francesco},
%  booktitle={Submitted},
%}
%
%@INPROCEEDINGS{Andrade,
%author={F. J. Andrade Chavez and S. Traversaro and D. Pucci and F. Nori},
%booktitle={2016 IEEE-RAS 16th International Conference on Humanoid Robots (Humanoids)},
%title={Model based in situ calibration of six axis force torque sensors},
%year={2016},
%pages={422-427},
%doi={10.1109/HUMANOIDS.2016.7803310},
%month={Nov},}


\subparagraph{Inferring the Operational Space and Appropriate Controls with Multiple Contacts (T4.2)}

Within this task...


\subparagraph{Generalizing and Improving Elementary Tasks with Contacts (T4.3)}

In this task, we aim to generate new skills from data, where elementary skills 
are acquired by imitation learning and transferred to novel situations using 
dynamic systems. During year one, TUD developed a novel representation of 
movement primitives that can be used for imitation learning from noisy observations.
Uncertainty of observed trajectories is explicitely modeled and used to generate new skills.
This movement representation has state-of-the-art capabilities in generalization, 
coupling between the degrees of freedom of the robot, and moreover, 
a time varying feedback controller can be derived in closed form. 
These features are partially illustrated in Figure \ref{fig:promps}.
This work was published
last year at the highly competitive conference on neural information processing \cite{Paraschos_NIPS_2013}.


In another work, published at the international conference on humanoid robots (HUMANOIDS), 
TUD demonstrated that this probabilistic approach for trajectory generation
has superior performance against deterministic policies. The use of
probability distributions over the trajectories increased significantly
 the generalization properties, which was evaluated on a high dimensional table
tennis scenario [Paraschos, A. and  Neumann, G and  Peters, J., 2013]. 
In the future work, we plan to incorporate external torque signals to initiate, 
maintain, and terminate contacts.

TUD also investigated how to learn human robot interaction through imitation. We presented a new approach to robot learning that allows anthropomorphic robots to learn a library of interaction skills from demonstration [H Ben Amor, D Vogt, M Ewerton, E Berger, B Jung and J Peters, 2014]. Traditional approaches to modeling interactions assume a pre-specified symbolic representation of the available actions. For example, they model interactions
in terms of commands such as \emph{wait}, \emph{pick-up}, and \emph{place}. Instead of such a top-down approach, we focused on learning responsive behavior in a bottom-up fashion using a trajectory based approach. The key idea behind our approach is that the observation of human-human collaborations can provide rich information specifying how and when to interact
in a particular situation. For example, by observing how two human workmen collaborate on lifting a heavy box, a robot could use machine learning algorithms to extract an
interaction model that specifies the states, movements, and situational responses of the involved parties. In turn, such a model can be used by the robot to assist in a similar lifting task. Our approach is as an extension of imitation learning to multi-agent scenarios, in which the behavior and the mutual interplay between two agents is imitated


We further extended the above approach by introducing \emph{Interaction Primitives} in [Ben Amor, H.; Neumann, G.; Kamthe, S.; Kroemer, O.; Peters, J., 2014]. Interaction primitives build on the framework of dynamic motor primitives (DMPs) by maintaining a distribution over the parameters of the DMP. With this distribution, we can learn the inherent correlations of cooperative activities which allow us to infer the behavior of the partner and to participate in the cooperation. A conceptual overview is sketched in Figure \ref{fig:interaction_primitives}. A learned Interaction Primitive can be used by a robot to (1) predict the human's next action in the current context, (2) identify the optimal response, (3) synchronize the movement with the human partner.

In the meantime, demonstration-based learning of "optimal trajectories" and stable controllers has been addressed by UPMC, in particular in \cite{stulp2013} where a general, flexible, and compact representation of parameterizable skills is proposed. This work generalizes the standard Dynamic Motor Primitive formulation in \cite{ijspeert2013} and proposes a novel DMP formulation for parametrized skills, based on additionally passing task parameters to the DMP function approximator. This generalizes previous approaches, in particular those which train and execute parametrized skills with two separate regressions. Learning the function approximator with one regression in the full space of phase and tasks parameters allows for more compact models, and the flexible use of different function approximator implementations such as LWPR and GPR, as we demonstrated on the Meka and iCub humanoids robots.
