%!TEX root = ../../secondYearReport.tex


\paragraph{Work package 4 progress}

\subparagraph{Improved Models from Real-Time Regression with Latent Contact Type Inference (T4.1)}%(TUD 8.5PM, ??PARTNERS)}

%IIT
Within T4.1 IIT developed a theoretical framework for estimating whole-body
dynamics from distributed multimodal sensors \cite{Nori2015}. Considered sensors
include joint encoders, gyroscopes, accelerometers and force/torque sensors.
Estimated quantities are position, velocity, acceleration and (internal and
external) wrenches on all the rigid bodies composing the robot articulated
chain. The estimation procedure consists of an extended Kalman filter (EKF)
which gives the a-posteriori estimation given all the available measurements.
Computational efficiency is obtained by formulating the Kalman filter
update-step with a sparse Bayesian network. Experiments for validating the
proposed theoretical framework have been conducted on a leg of the iCub humanoid
robot. The iCub is an ideal platform for the proposed experiment given its
distributed force, torque, linear acceleration and angular velocity sensors.
Results have shown the accuracy and the computational efficiency of the proposed
method. The theoretical framework has been implemented in an open source
software (see also Section \ref{sec:T15}).

%TUD
%LMProMPs and Model-free ProMPs
TUD extended their probabilistic movement primitive (ProMP) approach in two
ways. First, a mixture model approach that learns a shared latent structure of
related tasks from demonstrations was developed. The shared structure is encoded
by a multi-modal vector that modulates the probabilistic primitives. Both, the
probabilistic primitives as well as their activations (i.e., the latent
variable) are learned from demonstrations. In a table tennis ball prediction
tasks this latent variable modulated the slope and the waviness of the ball
trajectory. In a Kuka robot arm reaching task, the approach was used to learn
bi-modal reaching trajectories that avoid an obstacle placed in front of the
robot. This work is detailed in Deliverable D4.2 and will be presented at the
IEEE conference on Robotics and Automation in May in Seattle, USA
\cite{Rueckert_2015}. 

In a second extension of ProMPs, TUD developed a model-free control method that
can be trained from demonstrations and generates time-varying feedback control
gains that reproduces the demonstrations. In this approach a joint distribution
over states, sensory feedback (e.g., measured joint torques or contact forces)
and controls is learned. In conditioning on the current state the next-state
control-law can be computed in closed form approximating the true forward
dynamics through local linearizations given the demonstrations. TUD evalated
this model-free ProMP method on the humanoid robot iCub in lifting objects. A
conference paper is currently under review. 


\subparagraph{Inferring the Operational Space and Appropriate Controls with Multiple Contacts (T4.2)}%(TUD 4PM, ??PARTNERS)}

%TUD
For controlling high-dimensional robots, most stochastic optimal control
algorithms use approximations of the system dynamics and of the cost function
(e.g., using linearizations and Taylor expansions). These approximations are
typically only locally correct, which might cause instabilities in the greedy
policy updates, lead to oscillations or the algorithms diverge. To overcome
these drawbacks, TUD added a regularization term to the cost function that punishes
large policy update steps in the trajectory optimization procedure. TUD applied
this concept to the Approximate Inference Control method (AICO), where the
resulting algorithm guarantees convergence for uninformative initial solutions
without complex hand-tuning of learning rates. 

\begin{figure*}[t]
  \begin{center}
  \includegraphics[width=\textwidth]{./sections/WP4/pics_elmar/NaoReachingTask.png}
  \end{center}
  \caption{Reaching task with the humanoid robot \textit{Nao}. The robot has to
reach  a desired end effector position with the right arm while maintaining
balance. Eight snapshots of the inferred movement are shown in (A). In (B), the
convergence of the costs of the optimization procedure is shown, where we
compare \textit{iLQG}, the standard implementation of AICO and the regularized
variant. The movement objectives for the right arm are shown in the left panel
in (C). To balance the robot lifts its left hand and bends the head back.}
  \label{fig:naoReachingTask}
  \end{figure*}

The new algorithm was evaluated on
two simulated robotic platforms. A robot arm with five joints was used for
reaching multiple targets while keeping the roll angle constant. On the humanoid
robot Nao, we show how complex skills like reaching (see Figure \ref{fig:naoReachingTask}) and balancing can be
inferred from desired center of gravity or end effector coordinates. This work was published at 
the international conference on humanoid robots \cite{Rueckert2014}. 
Supplemental Matlab demo code is available online at http://www.ausy.tu-darmstadt.de/Team/ElmarRueckert.

\subparagraph{Generalizing and Improving Elementary Tasks with Contacts (T4.3)}%(TUD 6PM, ??PARTNERS)}

The advent of robots in our every day life can only be accomplished with
reliable mechanisms for movement generation.  Movement Primitives (MP) are a
well-established approach for representing modular and re-usable robot movement
generators that can be composed into complex movements.  An easy-to-learn
representation of the primitive is, additionally, the key of recent imitation
and reinforcement learning successes. Current MPs approaches offer viable
properties such as concise representations of the inherently continuous and
high dimensional space of robot movements, generalization capabilities to novel
situations, temporal modulation of the primitive, sequencing of primitives,
coupling between the degrees of freedom of the robot, and controllers for real
time execution. However, no single MP framework exists that offers all these
properties.  During year two, TUD extended previous results on modeling stochastic movements \cite{Paraschos2013,Paraschos2013a}, where a journal version is currently under review. 

\definecolor{light-gray}{rgb}{0.91,0.9,0.88}

\newcommand{\hockeyImLabel}[3]{%
\begin{tikzpicture}
\node[  anchor=south west,inner sep=0,%
        draw=gray,%
        %left color=gray,right color=white%
        %fill=light-gray%
] (image) at (0,0){
\includegraphics[width=0.29\textwidth]{#1}};
\begin{scope}[x={(image.south east)},y={(image.north west)}]
    %\draw[help lines,xstep=.1,ystep=.1] (0,0) grid (1,1);
    %\foreach \x in {0,1,...,9} { \node [anchor=north] at (\x/10,0) {0.\x}; }
    %\foreach \y in {0,1,...,9} { \node [anchor=east] at (0,\y/10) {0.\y}; }
    \node [fill=white,opacity=0.6,above right,font=\large] at (0.01,0.01) {#2};
    \node [fill=white,opacity=0.6,below left,font=\large] at (0.99,0.99) {#3};
\end{scope}
\begin{scope}[x={(image.south east)},%
              y={(image.north west)},% 
              on background layer]
    %\path[fill=light-gray] (0,0) rectangle (1,1);
    \path[outer color=light-gray,inner color=white] (0,0) rectangle (1,1);
    \draw[gray,opacity=0.15,xstep=.1,ystep=.1] (0,0) grid (1,1);
\end{scope}
\end{tikzpicture}%
}

\begin{figure*}
\centering
\hockeyImLabel{./sections/WP4/pics_alex/Setup_tr_sm.png}{$a$}{Setup}
\hockeyImLabel{./sections/WP4/pics_alex/Distances_tr_sm}{$b$}{Distance}
\hockeyImLabel{./sections/WP4/pics_alex/HockeyAngle_tr_sm}{$c$}{Angle}

\vspace{0.2em}

\hockeyImLabel{./sections/WP4/pics_alex/Joint_tr_sm}{$d$}{Union}
\hockeyImLabel{./sections/WP4/pics_alex/Combined_tr_sm}{$e$}{Combination}
\hockeyImLabel{./sections/WP4/pics_alex/Joint-LeftRight_tr_sm.png}{$f$}{Conditioning}

\caption{Robot Hockey. The robot shoots a hockey puck. We demonstrate ten straight
shots for varying distances and ten shots for varying angles. The
pictures show samples from the ProMP model for straight shots ($b$)
and angled shots ($c$). Learning from the union of the two data sets yields a model
that represents variance in both, distance and angle ($d$). Multiplying
the individual models leads to the combined a model that only reproduces shots
where both models had probability mass, in the center at medium distance
($e$). The last picture shows the effect of conditioning on only left
and right angles ($f$).}
\label{fig:Robot-Hockey}

\end{figure*}

TUD incorporated all the desirable properties 
current approaches offer into a single framework and, additionally, TUD 
introduced new operations on the primitives, such as continuous blending and
co-activation of multiple primitives.  Most importantly, in this approach, the
novel co-activation operator is capable of solving multiple tasks concurrently \ref{fig:Robot-Hockey}.
Furthermore, TUD's approach is capable of reproducing exactly the demonstrated
variability of the movement and the coupling between the degrees of freedom of
the robot.  In this approach, called Probabilistic Movement Primitives (ProMPs) \cite{Paraschos2013,Paraschos2013a},
TUD derived all operations in closed form. In order to use the ProMPs for online
feedback control, TUD also derived a stochastic feedback controller that
reproduces exactly the encoded primitive. TUD evaluated and compared this approach
on several simulated and real robot scenarios.

Probabilistic movement primitives are a promising approach for learning,
modulating, and re-using movements in a modular control architecture.  To
effectively take advantage of such a control architecture, ProMPs support
simultaneous activation, match the quality of the encoded behavior from the
demonstrations, are able to adapt to different desired target positions, and
efficiently learn by imitation. ProMPs meets all of the aforementioned
requirements.  The desired trajectory distribution of the
primitive is parametrized by a hierarchical Bayesian model with Gaussian distributions. The
trajectory distribution can be obtained from demonstrations and 
simultaneously defines a feedback controller which is used for movement
execution. Currently, TUD is investigating extensions of the ProMPs framework 
to tasks that involve 
contacts with the environment (see T4.1). In addition, TUD started to investigate the improvement of elementary skills encoded in ProMPs with 
reinforcement learning, where a conference paper was submitted for review.


\subparagraph{Learning the Prioritization of Tasks (T4.4)}%(TUD 3.2PM, INRIA 2PM)}

%Valerio's paper on learning activation policies in task controller.
In this task, we address the problem of learning the temporal profile of soft
task priorities and null-space projectors for the multi-task controllers
developed in WP3. Our preliminary results have been submitted to a robotics
conference\footnote{Modugno, V.; Neumann, G.; Rueckert, E.; Oriolo, G.; Peters,
J.; Ivaldi, S. \textit{Learning soft task prioirities and null-space projectors
for motion planning of redundant manipulators}. Submitted to IROS 2015.}.

The first controller of WP3, \textit{Prioritized Task-Space Inverse Dynamics},
is based on \textit{strict task hierarchies}, where a hierarchical ordering of
the tasks is set, such that critical tasks (or tasks that are considered as more
important) are fulfilled with higher priorities, and the low-priority tasks are
solved in the null-space of the higher priority tasks
\cite{DelPrete-2014-ID267}. The strict control approach requires the
pre-specification of the task hierarchy. However, in many contexts it is
difficult to organize the tasks in a stack and define their relative importance
in forms of priorities. When priorities are strict, a higher task can completely
block lower tasks, which can result in movements that are not satisfactory for
the robot mission (e.g., its ``global'' task). 

The second controller of WP3 is based on \textit{soft task hierarchies}, where
the solution is typically given by a combination of weighted tasks
\cite{Salini-2011-ID348}. The importance or ``soft priority'' of each individual
task is represented by a scalar weight. By tuning the vector of scalar weights,
evolving in time, the global robot behavior can be optimized. Within WP3, Liu et
al. \cite{liu_ICRA2015} propose a generalized projector (GHC) that handles
strict and non-strict priorities with smooth transitions when tasks priorities
are swapped. They show that adapting these weights may result in a seamless
transition between tasks (i.e., reaching for an object, staying close to a
resting posture and avoiding an obstacle) and in continuous task sequencing.
Despite the elegant framework, their controller needs again a lot of manual
tuning: particularly, the evolution of the tasks priorities in time, the timing
and the tasks transitions need to be designed by hand. While this approach could
still be easy for few tasks and simple robotic arms, it can quickly become
unfeasible for complex robots such as humanoids performing whole-body movements
that usually require a dozen of tasks and constraints (e.g., control balance,
posture, end-effectors, stabilize head gaze, prevent slipping, control
interaction forces etc.).

\begin{figure*}%[t!]
\centering
\includegraphics[width=.75\hsize]{./sections/WP4/pics_serena/concept_scheme}
\caption{This scheme briefly describes the proposed method. The control torques are computed by a combination of tasks weighted by soft priorities, represented as parameterized activation policies, that are multiplied by a Null-space projector, where some activation functions for different projectors are joined. The global task execution is evaluated and a fitness function is computed: a policy search method is then used to optimize the parameters of the activation policies, both for tasks and projector. }
\label{figure:scheme}
\end{figure*}

In this task we propose a first solution to the problem of how these weights can
be learned through trial-and-error. We study how the \textit{temporal} profiles
of the task weights can be learned from a reward function, which is assumed to
be given\footnote{For many robotic task, e.g., tracking desired center-of-mass
or end-effector trajectories while avoiding obstacles, such reward functions
have been defined in \cite{Kober_IJRR_2013}.}. 

As a first step towards a controller that is capable of handling multiple tasks
and constraints on a complex robot, while allowing us to efficiently learn the
task priorities, we propose a regularized version of the Unified Framework (RUF)
proposed by Peters et al \cite{Peters_AR_2008}, where the tasks weights and
Null space projectors weights are represented by parametrized functional
approximators that can be automatically determined through a stochastic
optimization procedure. The concept is presented in Figure~\ref{figure:scheme}.

\begin{figure}%[b!]

\begin{subfigure}{.3\textwidth}
  \centering
  \includegraphics[width=1\linewidth]{./sections/WP4/pics_serena/alpha1}
  \label{fig:alpha1}
  \caption{Attractor activation.}
\end{subfigure}%
\begin{subfigure}{.3\textwidth}
  \centering
  \includegraphics[width=1\linewidth]{./sections/WP4/pics_serena/alpha2}
  \label{fig:alpha2}
  \caption{Null-projector activation.}
\end{subfigure}
\begin{subfigure}{.38\textwidth}
  \centering
  \includegraphics[width=1\linewidth]{./sections/WP4/pics_serena/comparison}
  \label{fig:alpha2}
  \caption{Learning performance.}
\end{subfigure}
\caption{The panels in (a) and (b) show the mean and standard deviation of the temporal
profile of the activation functions $\alpha,\beta$, optimized by RUF+CMA-ES,
computed over $R=50$ replications of the same experiment of the table scenario.
(c) Comparison of our method (blue line) to the generalized projector method (GHC).}
\label{fig:activation_policy}
\end{figure}

As a first results, we show that the optimization process  generates weights
profiles that cannot be designed manually in advance, see panels (a) and (b) in 
Figure~\ref{fig:activation_policy}.

% \begin{figure}%[b!]
% \centering
% \includegraphics[width=1\hsize]{./sections/WP4/pics_serena/comparison}
% \caption{This figure shows the mean and the standard deviation of $R=20$
% replications of the same experiments for our method RUF+CMA-ES and GHC+CMA-ES.
% For both controllers, we consider a random initial point in the parameter space
% for the optimization algorithm. Our method shows a faster convergence learning
% rate and it reaches a better result in terms of the optimized fitness.}
% \label{fig:comparison}
% \end{figure}

We then compare the performance of our controller with the state-of-the-art
method GHC proposed by Liu et al. \cite{liu_ICRA2015} (WP3). We consider the
following experimental scenario: a 7-DOF KUKA Light Weight Robot arm, starting
from a vertical position, must reach a desired point with its end-effector,
while avoiding to collide with a table, represented by a surface parallel to the
z-axis in between the robot and the goal. The aim of the experiment is to bring
the end-effector as close as possible to the desired position, while avoiding
collisions with the obstacle. We define  $T=3$ tasks: a regulation task in the
joint space, and two reaching tasks for the elbow and the end-effector. For both
methods, we find the optimal profiles for the weighting functions with CMA-ES.
Our second result is that our controller generally performs better than GHC,
even if we optimize the policies in both methods: on an average of 20
replicates, our RUF+CMA-ES finds 90\% of the solutions found by our RUF+CMA-ES
satisfies the constraints, while only 75\% of the solutions of GHC+learning are
acceptable. Furthermore, the final best solution found by RUF+CMA-ES outperforms
the one of GHC+CMA-ES, as shown in Figure~\ref{fig:activation_policy} (c).

