%!TEX root = ../../thirdYearReport.tex


\newcommand{\EQ}{\!\!\!=\!\!\!}

\newcommand{\Bp}{\mathbf{p}}
\newcommand{\Br}{\mathbf{r}}
\newcommand{\Bf}{\mathbf{f}}
\newcommand{\BJ}{\mathbf{J}}
\newcommand{\Bv}{\mathbf{v}}
\newcommand{\BI}{\mathbf{I}}
\newcommand{\BR}{\mathbf{R}}
\newcommand{\BK}{\mathbf{K}}
\newcommand{\BD}{\mathbf{D}}
\newcommand{\BA}{\mathbf{A}}
\newcommand{\Bb}{\mathbf{b}}
\newcommand{\BM}{\mathbf{M}}
\newcommand{\BC}{\mathbf{C}}
\newcommand{\Bg}{\mathbf{g}}
\newcommand{\BS}{\mathbf{S}}
\newcommand{\Bzero}{\mathbf{0}}
\newcommand{\BN}{\mathbf{N}}
\newcommand{\Bh}{\mathbf{h}}
\newcommand{\BW}{\mathbf{W}}
\newcommand{\Bq}{\mathbf{q}}
\newcommand{\BF}{\mathbf{F}}
\newcommand{\Bn}{\mathbf{n}}

\newcommand{\Bomega}{\boldsymbol{\omega}}
\newcommand{\Btau}{\boldsymbol{\tau}}
\newcommand{\Balpha}{\boldsymbol{\alpha}}
\newcommand{\Bbeta}{\boldsymbol{\beta}}



\paragraph{Work package 1 progress}

\subparagraph{Software architecture design and evaluation of available
  open-source software pertinent to the scope of the project. (T1.1)}

The goal of T1.1 was to agree on a specific software architecture with
associated software tools whose specifications, dependencies and
interconnections meet the requirements and needs for achieving the goals of
the project.  The software, which is called \texttt{codyco-superbuild}, has
been available via github on
\texttt{https://github.com/robotology/codyco-superbuild} since the second year
of the project.  Details about the modules of the software are available in
deliverables D1.1, D1.2 and D1.3.

\subparagraph{Simulator for whole-body motion with contacts (T1.2)}

The CoDyCo project requires a modular, component-based dynamics simulation
software providing numerically stable, computationally efficient and
physically consistent simulations of whole-body virtual human(oid) systems in
contact with rigid or soft environments.  To this end, in year one, a new iCub
simulator was released and documented as part of deliverable D1.1.


\subparagraph{Control library for flexible specification of task space
  dynamics of floating base manipulators. (T1.3)}



\subparagraph{System dynamics estimation software. Extension to
environmental compliance estimation (T1.4)}

As a part of this task, UB and TUD developed a framework in order to estimate
the parameters of compliant contacts between the robot and its environment.
Using this framework allows us to predict contact forces in the next instant.
Assume that the body which is in contact with a soft surface is labeled with
$B_c$.  We characterize the contact surface of $B_c$ by $m$ fictitious contact
points on this body.  Let $\Bp_i$ denote the position of the i$^{th}$ contact
point ($i=1,2,\ldots,m$) in the world frame.  Therefore,
%
\begin{equation}
  \Bp_i = \Bp + \BR \Br_i \, ,
\end{equation}
%
where $\Bp$ is the position of the origin of the local frame of $B_c$ with
respect to the world frame, $\BR$ is the rotation matrix of $B_c$ with respect
to the world frame and $\Br_i$ is the position of i$^{th}$ contact point in
the local frame of $B_c$.  So
%
\begin{equation}
  \dot{\Bp}_i = \dot{\Bp} + \dot{\BR} \Br_i = \dot{\Bp} - (\BR\Br_i)^{\wedge}
  \Bomega = [ \BI_{3 \times 3} \; \; {}-(\BR\Br_i)^{\wedge} ] \BJ_s \Bv \, ,
  \label{pidot}
\end{equation}
%
where $\Bomega$ is the angular velocity of $B_c$ and $()^{\wedge}$ represents
the skew symmetric matrix.  We also have
%
\begin{equation}
  \ddot{\Bp}_i = \ddot{\Bp} + \ddot{\BR} \Br_i = \ddot{\Bp} - (\BR
  \Br_i)^{\wedge} \dot{\Bomega} + (\Bomega)^{\wedge} (\Bomega)^{\wedge} \BR
  \Br_i = [ \BI_{3 \times 3} \; \; {}-(\BR \Br_i)^{\wedge}] (\dot{\BJ}_s \Bv +
  \BJ_s \dot{\Bv}) + (\Bomega)^{\wedge} (\Bomega)^{\wedge} \BR \Br_i \, .
  \label{piddot}
\end{equation}
%

According to contact mechanics \cite{Johnson77} and its applications in
robotics \cite{Azad&Featherstone10, Azad&Featherstone14a, Hunt&Crossley75,
  Marhefka&Orin99}, there is a non-linear relationship between compliant
contact force and deformation and the rate of the deformation of the surface.
However, we can assume a locally linear relationship between the change of the
contact force and the change of the deformation and its rate.  Therefore,
according to this assumption, we can write
%
\begin{equation}
  \delta \Bf_i = \BK \delta \Bp_i + \BD \delta \dot{\Bp}_i \, ,
\end{equation}
%
where $\delta \Bf_i$ and $\delta \Bp_i$ are the changes of the contact force
and the deformation at the i$^{th}$ contact point, respectively, and $\BK$ and
$\BD$ are $3 \times 3$ matrices of the coefficients of stiffness and damping,
respectively.  By using a linear integration method, we can estimate $\delta
\Bp_i$ and $\delta \dot{\Bp}_i$ as
%
\begin{equation}
  \delta \Bp_i = \dot{\Bp}_i \delta t + \frac{1}{2} \ddot{\Bp}_i \delta t^2 \,
  ,
  \label{deltapi}
\end{equation}
%
and
%
\begin{equation}
  \delta \dot{\Bp}_i = \ddot{\Bp}_i \delta t \, ,
  \label{deltapidot}
\end{equation}
%
where $\delta t$ is the sampling time.  Hence, by substituting (\ref{pidot})
and (\ref{piddot}) into (\ref{deltapi}) and (\ref{deltapidot}), we will have
%
\begin{equation}
  \delta \Bf_i = \BA_i \dot{\Bv} + \Bb_i \, ,
\end{equation}
%
where
%
%
\begin{equation}
  \BA_i = \frac{1}{2} \delta t^2 [\BK \; \; -\BK(\BR \Br_i)^{\wedge}] \BJ_s +
  \delta t [\BD \; \; - \BD(\BR \Br_i)^{\wedge}] \BJ_s \, ,
\end{equation}
%
and
%
\begin{eqnarray}
  \nonumber \Bb_i & \EQ & \delta t [\BK \; \; -\BK(\BR \Br_i)^{\wedge}] \BJ_s
  \Bv + \frac{1}{2} \delta t^2 [\BK \; \; -\BK(\BR \Br_i)^{\wedge}]
  \dot{\BJ}_s \Bv\\ & & + \frac{1}{2} \delta t^2 \BK (\Bomega)^{\wedge}
  (\Bomega)^{\wedge} \BR \Br_i + \delta t [\BD \; \; -\BD(\BR \Br_i)^{\wedge}]
  \dot{\BJ}_s \Bv + \delta t \BD (\Bomega)^{\wedge} (\Bomega)^{\wedge} \BR
  \Br_i \, .
\end{eqnarray}
%
Also for the moment of the contact forces we have
%
\begin{equation}
  \delta \Bn_i = (\BR \Br_i)^{\wedge} \delta \Bf_i \, .
\end{equation}
%
Therefore, total change of force vector for $B_c$ will be
%
\begin{equation}
  \delta \Bf_s =
  \begin{bmatrix}
    \sum_{i=1}^m \BA_i \dot{\Bv} + \sum_{i=1}^m \Bb_i\\
    \\
    \sum_{i=1}^m (\BR \Br_i)^{\wedge} \BA_i \dot{\Bv} + \sum_{i=1}^m \Bb_i
  \end{bmatrix}
  =
  \begin{bmatrix}
    \BA_f \\
    \\
    \BA_n
  \end{bmatrix}
  \dot{\Bv} +
  \begin{bmatrix}
    \Bb_f \\
    \\
    \Bb_n
  \end{bmatrix}
  = \BA \dot{\Bv} + \Bb \, .
\end{equation}
%
Hence, the estimated (for the next instant) 6D force-torque vector of the
contact force of $B_c$ will be
%
\begin{equation}
  \hat{\Bf}_s = \Bf_s + \delta \Bf_s = \BA \dot{\Bv} + \Bb + \Bf_s \, .
\end{equation}
%
Note that for the constraints on the soft contact force (unilaterality and
friction cone), $\hat{\Bf}_s$ has to be expressed in the local frame of $B_c$
as $\hat{\BF}_s = \BR^T \hat{\Bf}_s$.

  
\subparagraph{Extension and enhancement of the iDyn library. (T1.5)}
\label{sec:T15}












