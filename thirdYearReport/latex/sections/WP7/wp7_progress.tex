%!TEX root = ../../thirdYearReport.tex


\paragraph{Work package 7 progress}

Dissemination and exploitation activities included the participation to international events addressed to both commercial and academic institutions. 

\subparagraph*{Dissemination activities towards academia, industry, and other users (T7.1)}

Dissemination activities were conducted thorough international publications, organisation of international events, talks at international conferences, press interviews and iCub expositions at international events. Here is the overall contribution subdivided by partner:

\begin{itemize}

\item IIT: 4 invited visiting periods, 4 international events participation, 12 publications (4 journal articles, 8 international conferences), several media coverage events.

\item TUD: 9 invited talks, 1 organised international events, 12 publications (2 journal articles, 10 international conferences), 3 media coverage events, 3 M.Sc. theses and one Ph.D. thesis. 

\item UPMC: 7 invited talks, 4 other events, 5 paper presentations at international Conferences, 1 international journal, 5 international conferences.

\item UB: 

\item JSI: 2 invited talks, 1 editorial for journal special issue, 3 publications (2 journal, 1 internal conferences), 0 media coverage events.

\item INRIA: 11 publications: 4 journals, 4 conference papers, 3 workshop papers, 2 international workshops: ICRA 2015 and BMVA 2015, 1 special issue organization in Autonomous Robots, dissemination of activies in several media.

\end{itemize}

Live demonstration of the iCub have been performed at several international and national events.  Some of these events were sponsored by CoDyCo and the following is a non exhaustive list:

\begin{enumerate}

\item Live video shooting at ``Italia's got talent'', italian national television show. Shooting: December 1st-3rd 2015. Location: Catanzaro, Italy. The iCub performed the CoDyCo demo based on whole-body torque controlled motions with switching motions.

\item Live demonstrations at the event: ``Italian Manufacturing Forum'', UIC Gleacher Center, Chicago, IL, March 30th, 2016. The iCub was shipped to Chicago to perform several iCub related demonstrations (e.g. iCub standing, iCub performing whole-body equilibrium tasks) at the Italian Manufacturing Forum. 

\end{enumerate} 

Among the invitations as a speaker at international events it is worth citing the following:

\begin{enumerate}

\item Francesco Nori invited talk at the dissemination event Creative mornings. Talk: Interacting with Humans with iCub-humanoid. Dates: Location: May 22nd 2015. Milano, Italy.

\item Francesco Nori invited talk at Convegno NanoItaly, Roma, 21-24 settembre 2015. Talk: Force and motion capture system
based on distributed micro-accelerometers, gyros, force and tactile sensing. Date: 21 settembre 2015.

\item Talk by Elmar Rueckert, 02/2015 Probabilistic Inference and Modeling of Human Motor Skill Learning. Invited Talk. Workshop with Marc Toussaint’s group, Wolfram Burgard’s group and Oliver Brock’s group, Manigod, France.

\item  V. Padois. Robotique industrielle: évolution, enjeux et perspectives, January 2016. Invited talk at CNER/SERECT.

\item  Ivaldi, S. (12/2015) Human-robot interaction with iCub. Invited talk at University of Plymouth, by Samantha Adams and Angelo Cangelosi.

\item Babic, Jan. Human-in-the-loop control of robots for industrial assembly tasks : invited talk, Omron Keihanna Technology Innovation Center, 7th September 2015, Kyoto, Japan.

\end{enumerate} 

Among the organised international events here is a non exhaustive list of the most relevant events:

\begin{enumerate}

\item The iCub Summer School, ``Veni Vidi Vici''. The iCub Summer School July 22-31, 2015. The school focuses on humanoid robotics and will host at least two iCub and a COMAN robot. 

\item ICRA 2015 Workshop ICRA 2015 Workshop ``Get in touch! Tactile \& force sensing for autonomous, compliant, intelligent robots'', May 30th 2015, Seattle (USA), organized by S. Ivaldi, L. Jamone and B. Siciliano (\url{http://www.ausy.tu-darmstadt.de/Workshops/ICRA2015TactileForce}). The workshop had 147 registered participants.

\item In July, during the RSS conference in Rome, a full day workshop titled ``Towards a Unifying Framework for Whole-body and Manipulation Control'' has been organised. Topics covered the following areas: contacts planning and control; whole-body task control; compliant whole-body movements; dynamics in humanoid robots; machine learning and optimization methods for contact planning and control.
\end{enumerate} 


