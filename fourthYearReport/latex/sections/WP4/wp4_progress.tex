%!TEX root = ../../fourthYearReport.tex


\paragraph{Work package 4 progress}

The progress for each task are described hereafter.

\subparagraph{Learning the Prioritization of Tasks (T4.4) (TUD: 4PM)}

TUD continued its research on learning task prioritizations from human demonstrations using probabilistic models. This work is currently under review and 
a draft of the paper was added to Deliverable D4.3 in Section 5.  Here is a short summary of the approach. 

Movement prioritization is a common approach
to combine controllers of different tasks for redundant robots.
Each task is assigned a priority, where either strict or 'soft'
priorities can be used. While movement prioritization is an
important concept in the control of whole body movements, it
has been less considered in learning-based approaches, where
prioritization allows us to learn different tasks for different
end-effectors, and subsequently reproduce an arbitrary, unseen
combination of these tasks. This paper combines Bayesian task
prioritization, a 'soft' prioritization technique, with probabilistic
movement primitives to prioritize full motion sequences.
Probabilistic movement primitives can encode distributions of
movements over full motion sequences and provide control
laws to exactly follow these distributions. The probabilistic
formulation allows for a natural application of Bayesian task
prioritization. We demonstrate how the 'soft' priorities can
be obtained from imitation learning and that our prioritized
learning architecture can reproduce unseen task-combinations.
Moreover, we require less data to learn a combination of tasks
than the traditional approach that directly models each task in
joint space. We evaluate our approach on reaching movements
under constraints with a redundant bi-manual planar robot
and the humanoid robot iCub. 

