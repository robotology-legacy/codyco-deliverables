%!TEX root = ../../fourthYearReport.tex

\paragraph{Work package 5 progress}

The activities in WP5 are divided into four tasks corresponding to the four years project duration. As a result, during the fourth year CoDyCo results concentrate on T5.4. The main result consist in the implementation of the validation scenario consisting of the balancing with the help of a caregiver. The main scientific contribution is described here \cite{latella2016whole}.

\subparagraph{Scenario 4: learning how to stand up with the help of a human caregiver (T5.4)}

The main contributions to T5.4 have been presented in ``Validation scenario 4: learning how to stand up with the help of a human caregiver'' which discusses the technical implementation of the fourth year validation scenario (see \url{https://github.com/robotology-playground/codyco-deliverables/tree/master/D5.4/pdf}). The software developed for the scenario implementation is released with an open-source license and distributed through github (\url{https://github.com/robotology/codyco}).

\subparagraph{Deviations from workplan}  

No deviations.

%\begin{itemize}
%\item[-] \emph{\color{red}[A summary of progress towards objectives and details for each task;]}
%\item[-] \emph{\color{red}[Highlight clearly significant results;]}
%\item[-] \emph{\color{red}[If applicable, explain the reasons for deviations from Annex I and their impact on other tasks as well as on available resources and planning;]}
%\item[-] \emph{\color{red}[If applicable, explain the reasons for failing to achieve critical objectives and/or not being on schedule and explain the impact on other tasks as well as on available resources and planning (the explanations should be consistent with the declaration by the project coordinator) ;]}
%\item[-] \emph{\color{red}[a statement on the use of resources, in particular highlighting and explaining deviations between actual and planned  person-months per work package and per beneficiary in Annex 1 (Description of Work);]}
%\item[-] \emph{\color{red}[If applicable, propose corrective actions.]}
%\end{itemize}
