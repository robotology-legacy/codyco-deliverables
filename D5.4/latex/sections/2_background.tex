%!TEX root = ../template.tex

\section{Background}
%
The aim of this Section is to provide a rapid fast-forward of what is the current
 direction of the scientific community on this topic.  Most of the
  studies on the pHRI take inspiration from the intrinsic behaviour of the human nature:
   the \emph{mutual adaptive nature} that automatically occurs when two
    humans are cooperating together to accomplish a common task.  
	\\
	\indent
To this purpose, the importance of understanding human dynamics goes without saying and it is 
a crucial aspect of current state-of-the-art studies. Since humans
 move by minimizing jerk trajectories \cite{Flash1985},
 a method based on the minimum jerk model is used as a suitable approximation 
 for estimating the human partner motions in \cite{Maeda2001}. Here the attempt is that of
  incorporating human
  characteristics in the control strategy of the robot. The weakness in  this type of approach,
   however, lies in the pre-determination of the task and in the role that the robot has to
    play in the task execution.
   Furthermore, the minimum jerk model reliability decreases considerably if the human partner
    decides to apply non-scheduled trajectory changes during the task \cite{Miossec2009}.
	  Another route for pHRI is the \emph{imitation learning} approach, 
	  where the movements of two 
	 human actors are typically retrieved with motion capture techniques, clustered in motion
	  database (\cite{Guerra2011}, \cite{Kuehne2011}, \cite{Wojtusch2015}) and then used to 
	 learn the interaction skills (\cite{Amor2014}, \cite{Tamim2014}, \cite{Tamim2016}).
%
%%%%%%%%%%%%%%%%%%%%%%%%%%%%%%%%%%%%%%%%%%%%%%%%%%%%%%%%%%%%%%%%%%%%%%%%%%%%%%%%%%%%%%%%%%%%%%%%
\subsection{Problem statement}
Unlike the current leaning, we want to pay more attention on the key role that 
a proper sensing technology for human beings together with dynamics estimation algorithms
  may offer for retrieving whole-body motions and interaction forces. 
 More in detail, our work will be based on the formalism adopted for humanoid robots by 
 making the assumption of modelling the human body as a articulated rigid multi-body system. 
The advantage of this choice is evident since it allows to handle both systems
with the same mathematical tools. In this domain, the application of the Euler-Poincar\'{e} 
formalism \cite{Marsden2010} leads to three sets of equations describing:
    $i)$ the motion of the robot,
	$ii)$ the motion characterizing the human,
	$iii)$ the linking equations characterizing the contacts between human and robot. 
%
\begin{eqnarray*}
\label{robot}
i) \hspace{0.6cm} 
\bm {\mathrm{M}}(\bm {{q}}) \dot{\bm{v}} + \bm {\mathrm{C}}(\bm {{q}},
 \bm{v})
\bm{v} + \bm {\mathrm{G}}(\bm {{q}}) = \begin{bmatrix} \bm 0 \\ {\bm \tau}
 \end{bmatrix} + \bm {\mathrm{J}}^\top(\bm {{q}}) \bm{\mathrm{f}}
\end{eqnarray*}
%
\begin{eqnarray*}
\label{human}
ii) \hspace{0.6cm}
\mathbb{M}(\bar{\bm q}) \dot{\bar{\bm v}} + \mathbb{C}(\bar{\bm q}, \bar{\bm v})
\bar{\bm v} + \mathbb{G}(\bar{\bm q}) = \begin{bmatrix} \bm 0 \\ {\bm \tau} \end{bmatrix}
+ \mathbb{J}^\top(\bar{\bm q}) \bm{\mathrm{f}}
\end{eqnarray*}
%
\begin{eqnarray*}
\label{linking}
iii) \hspace{0.4cm}
\begin{bmatrix} \bm {\mathrm{J}}(\bm {{q}}) & ~\mathbb{J}(\bar{\bm q}) \end{bmatrix}
	\begin{bmatrix} \dot{\bm{v}} \\ \dot{\bar{\bm v}} \end{bmatrix} +
		\begin{bmatrix} \bm {\dot{\mathrm{J}}}(\bm {{q}}) & 
			~\dot{\mathbb{J}}(\bar{\bm q})
		 \end{bmatrix}
			\begin{bmatrix}{\bm{v}} \\ {\bar{\bm v}} \end{bmatrix} = \bm 0
\end{eqnarray*}
% \begin{eqnarray*}
% \label{linking}
% iii) \hspace{1.7cm} \mathbb{L}(
% \bm{{q}},
% \bar{\bm q},
% \bm{\bm v},
% \bar{\bm v},
% \dot{\bm v},
% \dot{\bar{\bm v}},
%  ) = \bm 0.
% \end{eqnarray*}
%
\indent
Equations $i)$ and $ii)$ are floating base system representations of the
dynamics of the robot and  human models, respectively. 
Vectors $\bm {{q}}$ and $\bar{\bm q}$ represent the configuration space (i.e. the position and orientation of a chosen frame, called base frame, and the joints configuration) of the two systems. The velocity is represented by $\bm{v}$
 and $\bar{\bm v}$ for robot and human systems, respectively.
 The matrices $\mathrm{M}$,
  $\mathrm{C}$, $\mathrm{G}$ and $\mathbb{M}$, $\mathbb{C}$, $\mathbb{G}$ denote the 
  mass matrix, Coriolis matrix and the gravity bias term for the robot and the human 
  systems, respectively.  
   The forces the two systems
  exchange are denoted by $\bm{\mathrm{f}}$, which owns a proper dimension depending on the
  number of wrenches\footnote{As an abuse of notation, we define as wrench a quantity that is
   not the dual of a twist but a vector $\in \mathbb R^{6}$ containing both the forces and 
   the related moments.} exchanged during the interaction task\footnote{For the sake of simplicity,
  we omitted the forces the two systems exchange with the external environment 
  (i.e., the ground) from the 
  formulation of $i)$ and $ii)$. As a straightforward consequence, the linking
   equations between each system with the external environment are not considered.}. 
  The Jacobians associated with the forces $\bm{\mathrm{f}}$ are denoted by
   $\bm {\mathrm{J}}(\bm{\mathrm{q}}) $ and $\mathbb{J}(\bar{\bm q})$.  
   In $iii)$ we make the assumption of rigid contacts between the two systems.