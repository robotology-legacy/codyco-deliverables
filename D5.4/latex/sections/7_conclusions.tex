%!TEX root = ../template.tex

\section{Conclusions and Future Works}
%
One of the aim of the presented framework is the estimation of dynamic variables of 
a human being while is physically interacting with a robot.
 This paper is the extended version of the probabilistic framework explored in
  \cite{LatellaSensors2016} but we introduced here a more complex model ($23$ vs $3$
   links and $22$ vs $2$ joints) and we found that our framework is able to estimate
    human dynamics variables even in presence of a multi-DoFs model.  We performed
	 also a comparison with the well-established biomechanical software OpenSim:
	  the results were promising depicting a good bent of our framework in
 modelling possible inaccuracies in the model.
\\
\indent
In this paper the endeavour was to retrieve an estimation of the human dynamics by means of
 the MAP algorithm and to this purpose the robot was considered as a \emph{passive} forces
  measurer.  But since the human dynamics is of pivotal importance
   for a control design aimed at considering the \emph{human in the loop}, the forthcoming
    idea will be to provide in real-time the robot with the human force feedback
that could be used either as a tool for \emph{reactive} human-robot collaboration
(implying a robot reactive control) and, in a long-term perspective,
for \emph{predictive} collaboration, for enhancing remarkably the interaction naturalness.
Thus, in the near future a new robot controller has to be 
designed in order that the robot can adapt and adjust the interaction strategy accordingly.
 \\
 \indent
 In this work we applied the proposed approach to human models composed of 1-DoF revolute joints, 
 by using the classical formalism widespread in robotics \cite{Featherstone2008}.
 In particular we combined this type of joints to obtain a series of joints with a 
 high number of DoFs, that however are only a rough approximation of the complexity 
 exhibited by real-life biomechanical joints.
 While we chose this joint model for an initial exploration of the method, the proposed
  algorithm is not limited to this particular choice.
 In particular the properties of any joint (with an arbitrary number of DoFs) can be 
 encapsulated in an interface where the relative position, velocity and acceleration of 
 the two bodies connected by it are described by arbitrary functions of joint coordinates
  variables and their derivatives that can then be directly inserted in
   \eqref{eq:matRNEA} and \eqref{eq:measRNEA}. 
 In this way, any kind of joint modelling can be described under this formalism 
 (e.g., complex non-linear models, models of biomechanical joints based on tabulated data 
 obtained from human experiments \cite{delp1990interactive}). 
 A possible implementation is presented in
  \cite{seth2010minimal}, where it is used in the context of biomechanical simulations. 
 In the future we plan to translate this type of results in our framework, 
 to generalize our method to arbitrarily complex musculoskeletal models.
 
 
 
 
 
 
 
 
 
 
  
  