%% Template for EU deliverable, using the deliverable.sty style file

\documentclass[12pt,a4paper,twoside]{article}

%% common package
\usepackage[headers]{deliverable}
\usepackage{xspace}
\usepackage{verbatim}
\usepackage[usenames]{color}
\usepackage[usenames,dvipsnames]{xcolor}
\usepackage{graphicx}

\usepackage{url}
\usepackage{array}
\usepackage{amsmath,bm,amsfonts}
\usepackage{tikz}
\usetikzlibrary{arrows,automata}
\usepackage{IEEEtrantools}
\usepackage{mathtools}
\DeclareMathOperator*{\argmin}{argmin}
%
%
%\usepackage{times}
%
% numbers option provides compact numerical references in the text. 
\usepackage[numbers]{natbib}
\usepackage{multicol}
\usepackage[bookmarks=true]{hyperref}

\usepackage{graphicx,import}
\usepackage{mathtools, amssymb}
\usepackage{paralist}
\usepackage[pdf]{svg}
\usepackage{amsmath,amssymb,amsthm}

% \usepackage{mathptmx}      % use Times fonts if available on your TeX system
% insert here the call for the packages your document requires
%\usepackage{latexsym}
% etc.
% \usepackage{color}
\usepackage{graphicx} % for pdf, bitmapped graphics files
% \usepackage{float}
\usepackage{times}
% \usepackage{multicol}
% \usepackage{multirow}	% multi row for table
\usepackage{mathtools,amsfonts,amssymb,amsmath, bm}
% \usepackage{cases}
% \usepackage{enumerate}
\usepackage{caption}
\usepackage[font=small]{caption}
\usepackage{paralist}
% \usepackage{colortbl}
\usepackage{xparse}
\usepackage{float}
\usepackage{import}
\usepackage[mediumspace,mediumqspace,Grey,squaren]{SIunits}

%%%%%%%%%%%%%%%%%%%%%%%%%%%%%%%%%%%%%%%%%%%%%%%%%%%%%%%%%%%%%%%%%%%%%%%%%%%%%%%%%%%%%%%%%%%%%%%%
% please place your own definitions here and don't use \def but
% \newcommand{}{}
\newcommand{\fratop}[2]{\genfrac{}{}{0pt}{}{#1}{#2}}
\newcommand{\mx}[1]{\mathbf{\bm{#1}}} 				% Matrix symbol
\newcommand{\vc}[1]{\mathbf{\bm{#1}}} 					% Vector symbol
% \newcommand{\degree}{\ensuremath{^\circ}}				% define the degree symbol
\newcommand{\pder}[2]{\frac{\partial#1}{\partial#2}}		% partial derivative
\newcommand{\refframe}[1]{\mbox{\textless#1\textgreater}}	% to denote a reference frame
\DeclareMathOperator*{\argmax}{\arg\!\max}				% argmax
\DeclareMathOperator*{\dif}{\mathrm{d}}					% d
\DeclareMathOperator*{\half}{\frac{1}{2}}					% one half
\DeclareMathOperator{\sgn}{sgn}
\newcommand{\mat}[1]{\ensuremath{\begin{bmatrix}#1\end{bmatrix}}}	% matrix
\newcommand{\rank}[1]{\text{rank}\left(#1\right)}							% rank
\newcommand{\diag}[1]{\text{diag}\left(#1\right)}							% diag
\newcommand{\x}{\ensuremath{\times}}





% \pdfinfo{
%    /Author (Homer Simpson)
%    /Title  (Robots: Our new overlords)
%    /CreationDate (D:20160128120000)
%    /Subject ()
%    /Keywords (Robotics;)
% }

\newtheorem{assumption}{\bf{Assumption}}
\newtheorem{definition}{\bf{Definition}}
\newtheorem{boldLemma}{\bf{Lemma}}
%%

%%insert here other packages needed by sections

%%

%%%%%%%%%%%%%%%%%%%%%%%%%%%%%%%%%%%%%%%%%%%%%%%%%%%%%%%%%%%%%%%%%%%%%%%%%%%%%%
%%% Titlepage
%%%%%%%%%%%%%%%%%%%%%%%%%%%%%%%%%%%%%%%%%%%%%%%%%%%%%%%%%%%%%%%%%%%%%%%%%%%%%%

% declaration of variables used in style
\deliverableDocnumber{D5.4}
\deliverableTitle{Validation scenario 4: learning how to stand up with the help of a human caregiver.}

\deliverableAuthor{Francesco Nori}
\deliverableResponsiblePartner{IIT}
\deliverableAffiliation{% Insert here authors affiliations
 $^1$ IIT
}

\deliverableReviewer{Francesco Nori}
\deliverableCoordinator{Francesco Nori}
\deliverableActivityNumber{n} %% n=1,..,10
\deliverableActivity{RTD}
\deliverableDoctype{Deliverable} %% or Prototype
\deliverableClassification{Public} % or Consortium
\deliverableDistribution{Consortium} %
\deliverableStatus{Draft} % Draft or Final
\deliverableDeliveryDate{28/2/2016}
\deliverableFile{D5.4.pdf} % please do not use "-" in the name
\deliverableVersion{1.0}
\deliverableDate{Feb.~28, 2017}
\deliverableYear{2017}
\deliverablePages{\pageref{LastPage}}
\deliverableChangelog{v.1.0 & Feb 19, 2017 & First draft %%\\\hline
%%              v.2.0 & Feb 20, 2007 & Final version
}
\deliverableProjectStartingDate{1st March 2013}
\deliverableProjectEndDate{28th February 2017}
\deliverableProjectAcronym{CoDyCo}
\deliverableProjectTitle{Whole-Body Compliant Dynamical Contacts in Cognitive Humanoids}
 \deliverableContractNumber{600716}
 \deliverableProjectCoordinator{Istituto Italiano di Tecnologia}
 \deliverableProjectUrl{www.codyco.eu}
 \deliverableFrameworkProgramme{FP7}
 
 \deliverableWorkpackage{deliv WP5}
 \deliverableEditors{Daniele Pucci}
 \deliverableContributors{Daniele Pucci, Francesco Romano, Jorhabib Eljaik, Silvio Traversaro, Vincent Padois, Francesco Nori, Claudia Latella, Marta Lorenzini, Maria Lazzaroni}
 \deliverableReviewers{}
\deliverableAbstract{This deliverable discusses the technical details and choices for the implementation of the year-4 validation scenario of the CoDyCo project.  The validation scenario aims at verifying the control performances in the case the humanoid robot iCub must balance while interacting with humans. Physical human-robot interaction is a field of growing interest among the scientific community. One of the main challenges is to replicate the physical mutual interaction occurring during a collaborative task between humans. For this purpose, the knowledge about human whole-body motions and forces is mandatory but the current state of the art on robots ability in estimating them is not sufficient to yield to a suitable interaction. This paper is an extended version of our previous work and a first attempt to go into this direction inasmuch as a first human-robot interaction task was investigated. The results prove that our framework is able to estimate the human dynamics variables also in a context of human-robot interaction by laying the foundation for more complex collaboration scenarios. }
\deliverableReviewers{}
\deliverableKeywordList{Whole-body human dynamics, Human-robot physical collaboration, Probabilistic sensor fusion algorithm}

%%%%%%%%%%%%%%%%%%%%%%%%%%%%%%%%%%%%%%%%%%%%%%%%%%%%%%%%%%%%%%%%%%%%%%%%%%%%%%
%%% Sections
%%%%%%%%%%%%%%%%%%%%%%%%%%%%%%%%%%%%%%%%%%%%%%%%%%%%%%%%%%%%%%%%%%%%%%%%%%%%%%


%%
%%%%%%%%%%%%%%%%%%%%%%%%%%%%%% BEGIN DOCUMENT
\begin{document}

\deliverableMaketitle

%%TODO move to style
\newcolumntype{L}[1]{>{\raggedright\let\newline\\\arraybackslash\hspace{0pt}}m{#1}}
\newcolumntype{C}[1]{>{\centering\let\newline\\\arraybackslash\hspace{0pt}}m{#1}}
\newcolumntype{R}[1]{>{\raggedleft\let\newline\\\arraybackslash\hspace{0pt}}m{#1}}

\textbf{Document Revision History}
\begin{center}
\begin{tabular}{|C{2cm}|C{3cm}|p{5cm}|C{4cm}|}
\hline
\textbf{Version}&\textbf{Date}&\textbf{Description}&\textbf{Author}\\\hline
First draft & 19 Feb 2017 & In this version we simply write down a few considerations on the fourth year validation scenario as discussed after the mid-year CoDyCo meeting in Birmingham. & Daniele Pucci \\\hline
\hline
Final version & 27 Feb 2016 & None & Daniele Pucci \\\hline
\end{tabular}
\end{center}
 
 \clearpage

\newpage
\renewcommand*\contentsname{Table of Contents}
\renewcommand*\listfigurename{Index of Figures}
\tableofcontents
\newpage
\newpage

 

% \PACS{PACS code1 \and PACS code2 \and more}
% \subclass{MSC code1 \and MSC code2 \and more}

%%%%%%%%%%%%%%%%%%%%%%%%%%%%%%%%%%%%%%%%%%%%%%%%%%%%%%%%%%%%%%%%%%% SECTIONS


%%%==================================================================================================
%%% PROBLEM STATEMENT
%%%==================================================================================================

A fundamental problem for torque-controlled humanoid robots is to accurately model their dynamics in presence of contacts, e.g., during manipulation in clutter~\cite{Jain2013}, whole-body movements~\cite{Kajita2008} or ground contacts in locomotion~\cite{Calandra2014}.
Analytic models suffer from inaccurate dynamic parameters, unmodeled dynamics (e.g., friction, couplings, elasticities) and noisy sensor measurements.
With contacts, the problem is even more challenging, because of discontinuities and additional non-linearities, which are difficult to model or estimate.
%One particular reason for this challenge is that
%contacts cause non-linearities in the system dynamics, which
%are difficult to model analytically or estimate. 
Moreover, if contact locations are not fixed a priori or known with sufficient precision, small errors in the localization of the external force can substantially deteriorate the quality of the inverse dynamics~\cite{DelPrete2012}.
%Additionally, analytic models suffer from inaccurate dynamic parameters, unmodeled dynamics (e.g., friction, couplings, elasticities) and noisy sensor measurements.
%play in the joints => can be solved kinematically

Nevertheless, many modern control strategies like inverse dynamics control~\cite{Erez2012}, computed torque control~\cite{Siciliano2009} or model predictive control approaches~\cite{Naveau2014} rely on accurate dynamic models.
% too risky: there are techniques in NMPC that do not need this 
%(or even differentiable) dynamics models.
With inaccurate dynamics models these control strategies can produce suboptimal policies, by not taking the external forces (caused by contacts) into account, and even damages to the hardware.

%%%==================================================================================================
%%% OUR CONTRIBUTION
%%%==================================================================================================
%
\begin{wrapfigure}{r}{0.30\columnwidth}
%\begin{figure}[t]
	\vspace{-10pt}
	\centering
	\includegraphics[width=.999\linewidth]{robertoIROS/fig/iCubDarmstadt01_new}
	\caption{The humanoid robot \robot{} used in the experiments.}
    \vspace{-10pt}
	\label{fig:icub}
%\end{figure}
\end{wrapfigure}
%
%%-------------------------------------
%% Our approach
%%-------------------------------------
%
%As a first step toward a more informed controller that explicitly considers the effect of contacts, we propose to learn the dynamics model from tactile sensor readings and force-torque sensors. 
In contrast to classical techniques based on the identification of dynamics parameters~\cite{Yamane2011calibration,Ogawa2014,Traversaro2013}, we propose a fully data-driven machine learning approach based on non-parametric models, where both the rigid body dynamics as well as the effect of external forces on the robot structure are learned directly from data collected on the real robot.
The proposed model makes use of the raw sensory data and does not require a kinematic/dynamics calibration~\cite{Yamane2011calibration,Ogawa2014,Traversaro2013}: in particular, it does not need a spatially calibrated model of the skin~\cite{DelPrete2011}.
As a non-linear model for the inverse dynamics we propose to learn a ``mixtures of contacts'' based on Gaussian Processes (GP).
%Each of these GP experts models a single ``type'' of contact and can be easily learned.
%However, by using a gating network that activates and deactivates the individual GP experts, it is %possible to switch between contacts and generalize to more complex environments.
% \todo[inline, color=yellow]{I don't think this bit is necessary here...\\
% As ground-truth we use joint-torque sensor measurements,
% and  we compare to a state-of-the-art analytic modeling approach
% that assumes perfect knowledge of the contact location to model external forces.
% Our approach in contrast, uses simply the tactile information to activate the experts.}
% The Gaussian processes are trained with the robot configurations (in joint angles)
% and the force measurements of the joint-torque sensors.
%
%
%============================================================
% CHANGE ME
%

%%-------------------------------------
%% Experimental results
%%-------------------------------------
%

We evaluate our model learning approach on two different tasks using the arm of the \robot{} humanoid robot~\cite{Natale2013} (see \fig\ref{fig:icub}) and compare against a state-of-the-art analytic modeling approach.
%The learned inverse dynamics model outperforms the analytic approach and we demonstrate that the model can generalize to new environmental conditions, such as changing contact locations.
%To the best of our knowledge, this is the first demonstration of how joint torques can be learned on a humanoid robot equipped with tactile and force/torque sensors in presence of contacts.
%
% 
In the first task, the learned inverse dynamics is used to compensate for an unexpected obstruction and minimize the tracking error.
In the second task, we use the learned model on a controller designed to slide along an obstruction. 
The purpose of the sliding controller is to minimize the contact forces and therefore avoid to break the motors or the artificial tendons that actuate the joints in the case of unexpected contacts.



%!TEX root = ../template.tex

\section{Background}
%
The aim of this Section is to provide a rapid fast-forward of what is the current
 direction of the scientific community on this topic.  Most of the
  studies on the pHRI take inspiration from the intrinsic behaviour of the human nature:
   the \emph{mutual adaptive nature} that automatically occurs when two
    humans are cooperating together to accomplish a common task.  
	\\
	\indent
To this purpose, the importance of understanding human dynamics goes without saying and it is 
a crucial aspect of current state-of-the-art studies. Since humans
 move by minimizing jerk trajectories \cite{Flash1985},
 a method based on the minimum jerk model is used as a suitable approximation 
 for estimating the human partner motions in \cite{Maeda2001}. Here the attempt is that of
  incorporating human
  characteristics in the control strategy of the robot. The weakness in  this type of approach,
   however, lies in the pre-determination of the task and in the role that the robot has to
    play in the task execution.
   Furthermore, the minimum jerk model reliability decreases considerably if the human partner
    decides to apply non-scheduled trajectory changes during the task \cite{Miossec2009}.
	  Another route for pHRI is the \emph{imitation learning} approach, 
	  where the movements of two 
	 human actors are typically retrieved with motion capture techniques, clustered in motion
	  database (\cite{Guerra2011}, \cite{Kuehne2011}, \cite{Wojtusch2015}) and then used to 
	 learn the interaction skills (\cite{Amor2014}, \cite{Tamim2014}, \cite{Tamim2016}).
%
%%%%%%%%%%%%%%%%%%%%%%%%%%%%%%%%%%%%%%%%%%%%%%%%%%%%%%%%%%%%%%%%%%%%%%%%%%%%%%%%%%%%%%%%%%%%%%%%
\subsection{Problem statement}
Unlike the current leaning, we want to pay more attention on the key role that 
a proper sensing technology for human beings together with dynamics estimation algorithms
  may offer for retrieving whole-body motions and interaction forces. 
 More in detail, our work will be based on the formalism adopted for humanoid robots by 
 making the assumption of modelling the human body as a articulated rigid multi-body system. 
The advantage of this choice is evident since it allows to handle both systems
with the same mathematical tools. In this domain, the application of the Euler-Poincar\'{e} 
formalism \cite{Marsden2010} leads to three sets of equations describing:
    $i)$ the motion of the robot,
	$ii)$ the motion characterizing the human,
	$iii)$ the linking equations characterizing the contacts between human and robot. 
%
\begin{eqnarray*}
\label{robot}
i) \hspace{0.6cm} 
\bm {\mathrm{M}}(\bm {{q}}) \dot{\bm{v}} + \bm {\mathrm{C}}(\bm {{q}},
 \bm{v})
\bm{v} + \bm {\mathrm{G}}(\bm {{q}}) = \begin{bmatrix} \bm 0 \\ {\bm \tau}
 \end{bmatrix} + \bm {\mathrm{J}}^\top(\bm {{q}}) \bm{\mathrm{f}}
\end{eqnarray*}
%
\begin{eqnarray*}
\label{human}
ii) \hspace{0.6cm}
\mathbb{M}(\bar{\bm q}) \dot{\bar{\bm v}} + \mathbb{C}(\bar{\bm q}, \bar{\bm v})
\bar{\bm v} + \mathbb{G}(\bar{\bm q}) = \begin{bmatrix} \bm 0 \\ {\bm \tau} \end{bmatrix}
+ \mathbb{J}^\top(\bar{\bm q}) \bm{\mathrm{f}}
\end{eqnarray*}
%
\begin{eqnarray*}
\label{linking}
iii) \hspace{0.4cm}
\begin{bmatrix} \bm {\mathrm{J}}(\bm {{q}}) & ~\mathbb{J}(\bar{\bm q}) \end{bmatrix}
	\begin{bmatrix} \dot{\bm{v}} \\ \dot{\bar{\bm v}} \end{bmatrix} +
		\begin{bmatrix} \bm {\dot{\mathrm{J}}}(\bm {{q}}) & 
			~\dot{\mathbb{J}}(\bar{\bm q})
		 \end{bmatrix}
			\begin{bmatrix}{\bm{v}} \\ {\bar{\bm v}} \end{bmatrix} = \bm 0
\end{eqnarray*}
% \begin{eqnarray*}
% \label{linking}
% iii) \hspace{1.7cm} \mathbb{L}(
% \bm{{q}},
% \bar{\bm q},
% \bm{\bm v},
% \bar{\bm v},
% \dot{\bm v},
% \dot{\bar{\bm v}},
%  ) = \bm 0.
% \end{eqnarray*}
%
\indent
Equations $i)$ and $ii)$ are floating base system representations of the
dynamics of the robot and  human models, respectively. 
Vectors $\bm {{q}}$ and $\bar{\bm q}$ represent the configuration space (i.e. the position and orientation of a chosen frame, called base frame, and the joints configuration) of the two systems. The velocity is represented by $\bm{v}$
 and $\bar{\bm v}$ for robot and human systems, respectively.
 The matrices $\mathrm{M}$,
  $\mathrm{C}$, $\mathrm{G}$ and $\mathbb{M}$, $\mathbb{C}$, $\mathbb{G}$ denote the 
  mass matrix, Coriolis matrix and the gravity bias term for the robot and the human 
  systems, respectively.  
   The forces the two systems
  exchange are denoted by $\bm{\mathrm{f}}$, which owns a proper dimension depending on the
  number of wrenches\footnote{As an abuse of notation, we define as wrench a quantity that is
   not the dual of a twist but a vector $\in \mathbb R^{6}$ containing both the forces and 
   the related moments.} exchanged during the interaction task\footnote{For the sake of simplicity,
  we omitted the forces the two systems exchange with the external environment 
  (i.e., the ground) from the 
  formulation of $i)$ and $ii)$. As a straightforward consequence, the linking
   equations between each system with the external environment are not considered.}. 
  The Jacobians associated with the forces $\bm{\mathrm{f}}$ are denoted by
   $\bm {\mathrm{J}}(\bm{\mathrm{q}}) $ and $\mathbb{J}(\bar{\bm q})$.  
   In $iii)$ we make the assumption of rigid contacts between the two systems.
%!TEX root = ../template.tex

\section{Human Body Modelling}

We propose a human body reference model as an articulated multi-body skeleton with rigid
 bodies connected by 3 Degrees-of-Freedom (DoF) joints. Kinematic and dynamic properties 
 are defined as follows.
%
%%%%%%%%%%%%%%%%%%%%%%%%%%%%%%%%%%%%%%%%%%%%%%%%%%%%%%%%%%%%%%%%%%%%%%%%%%%%%%%%%%%%%%%%%%%%%%%%
\subsection{Kinematic properties}
Inspired by the biomechanical model developed for the Xsens MVN motion capture system
 \cite{Roetenberg2009} shown in Fig. \ref{fig:human_models}b, our model consists of a set of 
 23 rigid bodies with simple geometric
  shapes (parallelepiped, cylinder, sphere).  The origin of each link is located at 
  the parent joint origin, (i.e., the joint that connects the link to its
  parent). Figure \ref{fig:figs_human_JointLink}b shows  links and joints of the model.  
  The dimension of 
   each link is estimated by using data coming from motion capture acquisition.
%
\begin{figure}
  \centering
    \includegraphics[width=1\columnwidth]{figs/human_JointLink}
\caption[Caption for LOF]{(a) Sensor attached to a generic link. 
(b) Human body reference model with labels for links and joints 
and with sensors distributed in the Xsens suit.  
 Reference frames are also shown\footnotemark.}
  \label{fig:figs_human_JointLink}
\end{figure}
%
\footnotetext{The RGB (Red-Green-Blue) convention for
${x}$-${y}$-${z}$ axes is adopted throughout the paper.}
%
%%%%%%%%%%%%%%%%%%%%%%%%%%%%%%%%%%%%%%%%%%%%%%%%%%%%%%%%%%%%%%%%%%%%%%%%%%%%%%%%%%%%%%%%%%%%%%%%
\subsection{Dynamic properties}
The dynamic properties, such as center of mass and inertia tensor for each link, are not
 embedded in the Xsens output data since they are usually computed in a post-processing phase.
   Since our aim is to have a real-time estimation for the human dynamic variables,
    the knowledge of
    dynamic properties during the acquisition phase is mandatory \cite{Drillis1964}.  Since it is impractical to retrieve these quantities in-vivo for humans, we relied on the
	 available anthropometric data in literature (\cite{Winter1990}, \cite{Herman2007})
	  starting from the total body mass of the
 subject, under the assumptions of geometric approximation and of homogeneous density for
  the rigid bodies (\cite{Hanavan1964}, \cite{Yeadon1990}).  
 %  
%%%%%%%%%%%%%%%%%%%%%%%%%%%%%%%%%%%%%%%%%%%%%%%%%%%%%%%%%%%%%%%%%%%%%%%%%%%%%%%%%%%%%%%%%%%%%%%%
%  \subsection{Sensor position}
% In order to build the model, the information of the position of the sensors for each link is
%  needed. Since it is not provided by Xsens, we exploited the linear acceleration $\bm a$ and
%   the angular velocity $\bm \omega$ measured by sensors in the following way:
%
% \begin{eqnarray} \label{eq:sensorPosEq} \nonumber
% {}^{S} \bm a_{S} &=& {}^{S} \bm R_G \Big( {}^{G} \bm a_{S} - {}^{G} \bm a_g \Big) =\\\nonumber
%                  &=& {}^{S} \bm R_G \Big[ {}^{G} \bm a_{L} + {}^{G} \bm{\dot \omega}_{L}
% 				     \times {}^{G} \bm R_{L} {}^{L} \bm p_{S, L}+\\
%                 &{}& {}^{G} \bm \omega_{L} \times \Big( {}^{G} \bm \omega_{L} \times
% 				     {}^{G} \bm R_{L} {}^{L} \bm p_{S, L}\Big) - {}^{G} \bm a_g \Big],
% \end{eqnarray}
% where $S$ is the frame of the sensor and $L$ is the frame related to the link on which it is
% attached (as in Fig. \ref{fig:figs_human_JointLink}a). By exploiting \eqref{eq:sensorPosEq}
% it is possible to retrieve the position of sensor on the link in
% the link frame ${}^{L} \bm p_{S, L}$.
%!TEX root = ../template.tex

\section{Probabilistic Sensor Fusion Algorithm}
%
In this Section we briefly recall the probabilistic method for estimating dynamic variables of
 an articulated mechanical system by exploiting the so-called sensor fusion information,
  already presented
  in our previous work (the reader should refer to \cite{LatellaSensors2016} for a more thorough
   presentation).
\\
\indent
From a theoretical point of view, we describe our model
 as a mechanical system represented by an oriented kinematic tree with $N_B$ moving links and
  $n$-DoFs.  Note that $n=n_1+...+n_{N_B}$ is the total number of DoFs of the system.  The
   generic $i$-th link  and its parent are coupled with a joint $i$ following 
  the topological Denavit-Hartenberg convention for joint numbering \cite{Denavit1955}.
We are interested in computing an estimation of a vector of \emph{dynamics variables} $\bm d$ defined as:
%
\begin{subequations} \label{eq:dynvec} 
	\begin{eqnarray*} 
    	\bm d &=& \begin{bmatrix} \bm d_{1}^\top & \bm
    	d_{2}^\top & \hdots & \bm d_{N_B}^\top \end{bmatrix}^\top 
    	\in \mathbb R^{24N_B+2n},\\
    \bm d_i &=& \begin{bmatrix} \bm
	a_{i}^\top &{\bm f_{i}^B}^\top & \bm f_{i}^\top & \bm \tau_{i} 
	& {\bm f_{i}^x}^\top & \ddot {\bm{q}}_{i} \end{bmatrix}^\top \in
	\mathbb R^{24+2n_i},
	\end{eqnarray*} 
\end{subequations}
%
where $\bm a_i$ is the $i$-th body spatial acceleration, $\bm {f_i}^B$ is the net
 wrench, $\bm f_i$ is the internal wrench exchanged from the parent link to the $i$-th link, 
 $\bm \tau_i \in \mathbb R^{n_i}$ is the torque at the joint, $\bm {f_i}^x$ is the external
  wrench applied by the environment to the link and $ \ddot{\bm q}_i  \in \mathbb R^{n_i}$ is
   the joint acceleration. The system can interact with the surrounding environment, and the
    result of this interaction is reflected in the presence of the external wrenches 
	$\bm {f_i}^x$.
	\\
	\indent
The dynamics of the mechanical system\footnote{We consider here the fixed base system
 configuration.} can be obtained from the application of the Newton-Euler
 equations\footnote{It is worth to notice that here we prefer to adopt the Newton-Euler
  formalism as an equivalent representation of the system dynamics. More details about 
  this choice in Section 3.3 of \cite{LatellaSensors2016}.} \cite{Featherstone2008}.  
  It is possible to rearrange these equations
  into a matrix form thus obtaining the following linear system of equations in the variable
    $\bm d$:
%
\begin{equation} \label{eq:matRNEA} 
\bm D(\bm q, \dot {\bm q}) \bm d + \bm b_D 
(\bm q,\bm {\dot q})= \bm 0,
\end{equation}
where the matrix $\bm D \in \mathbb R^{(18 N_B+n) \times \bm d}$  and the bias vector
$\bm b_D \in \mathbb R^{18 N_B+n}$.  We now consider the presence of $N_S$ measurements of dynamic quantities coming from different
 sensors (e.g. accelerometers, force/torque sensors) and we denote with $\bm y \in \mathbb
  R^{N_S}$ the vector containing all the measurements.  The dynamic variables and the values
   measured by the sensors can be related by the following set of equations:
%
\begin{equation} \label{eq:measRNEA} 
\bm Y(\bm q, \dot {\bm q}) \bm d + \bm b_Y (\bm q,\bm {\dot q})= \bm y,
\end{equation}
where $\bm Y \in \mathbb R^{N_S \times \bm d}$ and $\bm b_Y \in \mathbb R^{N_S}$.  By stacking
 together \eqref{eq:matRNEA} and \eqref{eq:measRNEA} we obtain a linear system of
  equations in the variable $\bm d$:
%
\begin{eqnarray} \label{eq:systemEq} 
\begin{bmatrix}
\bm Y(\bm q,\dot {\bm q}) \\ \bm D(\bm q, \dot {\bm q}) \\
 \end{bmatrix}\bm d + \begin{bmatrix} \bm b_Y(\bm q, \dot {\bm q})
\\ \bm b_D(\bm q, \dot {\bm q}) \end{bmatrix} = \begin{bmatrix} \bm y\\
 \bm 0 \end{bmatrix}. 
\end{eqnarray}
%
\indent
Equation \eqref{eq:systemEq} describes, in general, an overdetermined linear system of
 equations.  The bottom part, corresponding to \eqref{eq:matRNEA} represents the
  Newton-Euler equations, while the upper part contains the information coming from the,
   possibly noisy or redundant, sensors.
It is possible to compute the whole-body dynamics estimation
by solving the system in \eqref{eq:systemEq} for $\bm d$.  One possible approach is to solve \eqref{eq:systemEq} in the
 least-square sense, by using a Moore-Penrose pseudoinverse or a weighted pseudo-inverse.  
 
 In the following we perform a different choice.  
  We frame the estimation of $\bm d$ given the knowledge of $\bm y$ and prior information 
  about the model and the sensors in a Gaussian domain by
   means of a \emph{Maximum-a-Posteriori} (MAP) estimator\footnote{The benefits of the MAP
    estimator choice are explained in Section 4 of \cite{LatellaSensors2016}.} such that
%
\begin{equation*} 
\bm d_{\mbox\footnotesize{MAP}}=\arg \max_{\bm d} p(\bm d| \bm y) .
\end{equation*}
Since in this framework probability distributions are associated to both the measurements
 and the model, it suffices to
 compute the expected value and the covariance matrix of $\bm d$ given $\bm y$, i.e.
%
\begin{subequations}\label{eq:d_y} 
	\begin{eqnarray}\label{eq:d_ySigma} 
	 \bm {\Sigma}_{d|y}& = & \left(\bar{\bm {\Sigma}}_D^{-1}+ \bm Y^\top \bm 
	 {\Sigma}_{y}^{-1} \bm Y \right)^{-1},\\ \label{eq:d_yMu} 
	 \bm {\mu}_{d|y} &= & \bm {\Sigma}_{d|y} \left[ \bm Y ^\top \bm {\Sigma}_{y}^{-1} 
	 (\bm y - \bm b_Y) + \bar{\bm {\Sigma}}_D^{-1} \bar {\bm {\mu}}_D \right],
    \end{eqnarray} 
\end{subequations}
%
where $\bar{\bm {\mu}}_D$ and $\bar{\bm {\Sigma}}_D$ are the mean and 
covariance of the probability distribution 
 $\small{p(\bm d) \sim \mathcal N \left(\bar  
{\bm {\mu}}_D, \bar {\bm {\Sigma}}_D\right)}$ of the model, respectively;
 $\bm {\Sigma}_{y}$ is the covariance 
matrix of the distribution $\small{p(\bm y)\sim\mathcal N \left({\bm {\mu}}_y, {\bm {\Sigma}}_y\right)}$ related to the measurements.
  In the Gaussian framework, \eqref{eq:d_yMu} 
corresponds to the estimation of $\bm d_{\mbox\footnotesize{MAP}}$.  
It is worth noting that the vector $\bm d$ contains, among the other dynamic variables, an estimate of the joint torque $\bm \tau$ for retrieving the inverse dynamics estimation.
%!TEX root = ../D5.4.tex

\section{Experimental Design}
%
In this experiment the iCub is torque controlled. The control algorithm relies on the inverse-dynamics 
control scheme that was presented in \cite{NoriTrav2015}. Human dynamics and kinematics are monitored by 
whole-body distributed IMU sensors and contact forces at the feet are measured with two force platforms.
%
%%%%%%%%%%%%%%%%%%%%%%%%%%%%%%%%%%%%%%%%%%%%%%%%%%%%%%%%%%%%%%%%%%%%%%%%%%%%%%%%%%%%%%%%%%%%%%%%
\subsection{Human wearable sensors for dynamic estimation}
 Human kinematics data were acquired by using a full-body wearable lycra suit provided by Xsens Technologies.  
The wearable suit is composed of 17 wired trackers, (i.e., inertial sensor units-IMUs including an
 accelerometer, a gyroscope and a magnetometer). The suit has signal transmitters that send
  measurements to the acquisition unit through a wireless receiver which collects data at a
   frequency of \unit{240}{\hertz}. The human subject performed the required
    task standing with the
   feet on two standard force platforms AMTI OR6 mounted on the ground, while interacting 
   with the robot.
	  Each platform acquired a wrench sample at a frequency of
	   \unit{1}{\kilo\hertz} by using AMTI acquisition units. 
%
\begin{figure}
  \centering
    \includegraphics[width=0.9\columnwidth]{figs/humanModels}
  \caption{(a) Subject with the motion capture suit. (b) The Xsens MVN model. (c) Model reconstructed in OpenSim by using virtual markers from Xsens acquisition.}
 \label{fig:human_models}
\end{figure}
%
%%%%%%%%%%%%%%%%%%%%%%%%%%%%%%%%%%%%%%%%%%%%%%%%%%%%%%%%%%%%%%%%%%%%%%%%%%%%%%%%%%%%%%%%%%%%%%%%
\subsection{Robot sensors for dynamic estimation}
Experiments were conducted on the iCub \cite{Metta2010}, a full-body humanoid
robot (Fig. \ref{fig:iCub_couple}a) with 53-DoFs: 6 in the head, 16 in each arm, 3 in the
 torso and 6 in each leg. The iCub is endowed with whole-body distributed force/torque sensors,
  accelerometers, gyroscopes and tactile sensors. Specifically, the limbs are equipped with six
   force/torque sensors placed in the upper arms, in the upper legs and in the ankles 
   (Fig. \ref{fig:iCub_couple}b). Internal joint torques and external wrenches are estimated
    through an online whole-body estimation algorithm \cite{Nori2015icub}. Measurements for the
    wrenches exchanged between the robot and the human are obtained thanks to it.  Robot data
	 were collected at a frequency of \unit{100}{\hertz}.
%
\begin{figure}
  \centering
    \includegraphics[width=0.55\columnwidth]{figs/iCub_couple}
  \caption{(a) The humanoid iCub. (b) Model of the iCub with the force/torque 
  sensors embedded in the limbs structure.}
  \label{fig:iCub_couple}
\end{figure}

\begin{figure}
  \centering
    \includegraphics[width=0.45\columnwidth]{figs/iCubStool1} \hspace{1cm}
    \includegraphics[width=0.45\columnwidth]{figs/iCubStool2}
  \caption{The iCub is initially seating on a stool while keeping the feet on the ground. }
  \label{fig:iCubStool}
\end{figure}
%
%%%%%%%%%%%%%%%%%%%%%%%%%%%%%%%%%%%%%%%%%%%%%%%%%%%%%%%%%%%%%%%%%%%%%%%%%%%%%%%%%%%%%%%%%%%%%%%%
\subsection{Procedure protocol}
The interacting subject (e.g. caregiver) wears the suit (Fig. \ref{fig:human_models}a) and stands on the 
two force plates by positioning each foot on a platform. The robot 
is located in front of the subject, facing him and  seating on a stool.
It maintains balance with the whole-body inverse dynamics approach described in \cite{NoriTrav2015}.
During the experiment, human and robot interact by exchanging forces at predefined locations. 
At this stage, the interaction was chosen to occur at the iCub forearm to avoid mechanical failures
due to the fragility of the iCub hands  (as shown in Fig. \ref{fig:interaction_lateral&top}a).
Also the relative human-robot distance
is fixed by requiring the human subject to place the feet at specific locations on a printed paper
which sketches the experiment layout and is placed on the floor (Fig. \ref{fig:interaction_lateral&top}b). 

The basic control strategy for the standing motion relies on whole-body inverse dynamics.
The controller is implemented in Simulink\footnote{\url{https://github.com/robotology-playground/WBI-Toolbox-controllers}}
and has been successfully tested in 
simulation (see Fig.~\ref{fig:iCubSimStanding}) and on the real robot (see Fig.~\ref{fig:iCubStanding}).
In its current version the controller performs the standing up motion without the help
of the caregiver, i.e. without any physical human-robot interaction. 

  
%
\begin{figure}[ht]
  \centering
    \includegraphics[height=6cm]{figs/interaction_lateralAndTop2} \hspace{1cm}
    \includegraphics[height=6cm]{figs/interaction_lateralAndTop}
          \caption{(a) The figure shows the
		  reference frames for the force/torque sensor of the robot (iCubFT), the robot fixed
		   base (iCubFB), the force plate (FP), the human fixed base (hFB), the human foot and
		    hand (hFOOT, hHAND) respectively. (b) Top view for the feet position layout.}
			\label{fig:interaction_lateral&top}
\end{figure}

\begin{figure}
  \centering
    \includegraphics[width=0.2\columnwidth]{figs/standingSim_1} \hspace{1cm}
    \includegraphics[width=0.2\columnwidth]{figs/standingSim_2} \hspace{1cm}
    \includegraphics[width=0.20\columnwidth]{figs/standingSim_3}
  \caption{The iCub simulated standing motion without external support from a caregiver. }
  \label{fig:iCubSimStanding}
\end{figure}

\begin{figure}
  \centering
    \includegraphics[width=0.25\columnwidth]{figs/standing_1} \hspace{1cm}
    \includegraphics[width=0.25\columnwidth]{figs/standing_2} \hspace{1cm}
    \includegraphics[width=0.25\columnwidth]{figs/standing_3}
  \caption{The iCub standing motion without external support from a caregiver. }
  \label{fig:iCubStanding}
\end{figure}
%!TEX root =  ../D5.4.tex

\section{Experimental Results}
%
In this Section we discuss the evaluation procedure to prove the computation effectiveness of
 our estimation algorithm. The analysis was mainly performed on Matlab. 
 %and the related code 
%is freely available on Github\footnote{\texttt{https://github.com/claudia-lat/MAPest}}.
%
%%%%%%%%%%%%%%%%%%%%%%%%%%%%%%%%%%%%%%%%%%%%%%%%%%%%%%%%%%%%%%%%%%%%%%%%%%%%%%%%%%%%%%%%%%%%%%%%
\subsection{Torques estimation}
The core of the experiment consists in estimating the
 variable $\bm d$.  Among the variables contained in $\bm d$, the quantities of major 
 interest in our analysis are 
 the torques $\bm{\tau}$. We consider the internal
 torques developed along the $y$ axis in which the most significant angle variation is observed. 
In particular, we take into account the torque at the hip for $BT$ and the knee for $ST$. Since
 the torque estimation provides qualitatively a comparable result for both the two sides of 
 the human body, it is exhaustive to show only the torques associated to one side, e.g. the
  right one.
The value of the torque mostly depends on the kinematics and further on the 
inertial parameters of the subject and therefore, in order to compare torques across different subjects,
 it has to be normalized by considering the maximum and minimum values of each 
 subject's torque. 
Figures \ref{fig:figs_torqueRightHipBH}-\ref{fig:figs_torqueRightHipBRA} show the mean 
and the standard deviation of the right hip $\bm{\tau}$ estimation  without and 
with the interaction with the robot, respectively.  Figures \ref{fig:figs_torqueRightKneeSH}-\ref{fig:figs_torqueRightKneeSR} show the same 
quantities for the torques at the right knee in a $ST$ task.
%
% \begin{figure}[h!]
%   \centering
%    \begin{subfigure}[b]{0.49\columnwidth}
%       \includegraphics[width=\textwidth]{figs/torqueRightHipBH.pdf}
%           \caption{}
%           \label{fig:figs_torqueRightHipBH}
%   \end{subfigure}
%    \begin{subfigure}[b]{0.49\columnwidth}
%     \includegraphics[width=\textwidth]{figs/torqueRightHipBRA.pdf}
% 	\caption{}
%         \label{fig:figs_torqueRightHipBRA}
%    \end{subfigure}
%    \begin{subfigure}[b]{0.49\columnwidth}
%     \includegraphics[width=\textwidth]{figs/torqueRightKneeSH.pdf}
% 	\caption{}
%         \label{fig:figs_torqueRightKneeSH}
%    \end{subfigure}
%    \begin{subfigure}[b]{0.49\columnwidth}
%     \includegraphics[width=\textwidth]{figs/torqueRightKneeSR.pdf}
% 	\caption{}
%         \label{fig:figs_torqueRightKneeSR}
%    \end{subfigure}
%           \caption{\emph{Inter-subjects analysis}: normalised right hip torques of 10 subjects
% 		  (mean and standard deviation) for BT without (a) and with (b) robot, for ST without
% 		   (c) and with (d) robot.}
% \end{figure}
%
 \begin{figure*}[!ht]
	 \centering
	%\begin{subfigure}[b]{0.48\textwidth}
		\includegraphics[width=0.48\textwidth]{figs/torqueRightHipBH}
	%	\caption{}
	%	\label{fig:figs_torqueRightHipBH}
	% \end{subfigure}
 %	\begin{subfigure}[b]{0.48\textwidth}
 		\includegraphics[width=0.48\textwidth]{figs/torqueRightHipBHR}
 %		\caption{}
 %		\label{fig:figs_torqueRightHipBRA}
 %	 \end{subfigure}
 %	\begin{subfigure}[b]{0.48\textwidth}                                                                 
   \includegraphics[width=0.48\textwidth]{figs/torqueRightKneeSH}
	%	 \caption{}
	%	\label{fig:figs_torqueRightKneeSH}
 %	 \end{subfigure}
 %	\begin{subfigure}[b]{0.48\textwidth}                                                         
 \includegraphics[width=0.48\textwidth]{figs/torqueRightKneeSHR}
 %		  	\caption{}
 	%	  	 \label{fig:figs_torqueRightKneeSR}
  %	 \end{subfigure}
      \caption{\emph{Inter-subjects analysis}: normalized torques of 10 subjects
		  (mean and standard deviation) for right hip in $BT$ without (a) and with (b) 
		  robot, for right knee in $ST$ without (c) and with (d) robot.}
 \end{figure*}
%
%%%%%%%%%%%%%%%%%%%%%%%%%%%%%%%%%%%%%%%%%%%%%%%%%%%%%%%%%%%%%%%%%%%%%%%%%%%%%%%%%%%%%%%%%%%%%%%%
\subsection{Robustness test}
To test the robustness of the method with respect to modeling errors we asked one subject to
perform the $BT$ with the robot in two different configurations, i.e. \emph{with} and
 \emph{without} an additional mass ($W$) of \unit{6}{\kilo}{\gram} roughly positioned in
  correspondence of the torso center of mass. The MAP computation was performed by
   considering as algorithm inputs the following cases (see Table \ref{table:robustness}):
	\begin{itemize}
		\item \emph{case A}: model of the subject without $W$ and measurements acquired 
		while performing the $BT$ with $W$;
		\item \emph{case B}: model of the subject without $W$ and measurements acquired 
		while performing the $BT$ without $W$.
		\end{itemize} 
Since in both the cases the analysis is performed with the model  of the subject without $W$, in
order to highlight a lower reliability for the model used in 
\emph{case A} computation, it is assigned a value to the model variance
\footnote{We refer here to the model covariance associated to the model as a diagonal matrix 
where each element of the diagonal is the variance value.}
 equal to $10^{-1}$ 
(different from the value of variance equal to $10^{-4}$ assigned for the \emph{case B}).
%
\begin{table}[H]
\caption{Cases for the MAP evaluation.}
\label{table:robustness}
\centering
\footnotesize
   \begin{tabular}{ l|| l | l | l | l | l | l | l | l  } 
    \multicolumn{1}{c||}{} &
    \multicolumn{3}{c|}{\emph{\textbf{case A}}} &
      \multicolumn{3}{c|}{\emph{\textbf{case B}}}\\
    \hline
	\hline
        \multicolumn{1}{|c||}{\emph{\textbf{model}}} &
    \multicolumn{3}{c|}{without W} &
      \multicolumn{3}{c|}{without W} \\
    \hline
    \multicolumn{1}{|c||}{\emph{\textbf{measurements}}} &
    \multicolumn{3}{c|}{with W} &
      \multicolumn{3}{c|}{without W}\\
	\hline
      \multicolumn{1}{|c||}{$\Sigma$ \emph{\textbf{model}}} &
      \multicolumn{3}{c|}{$10^{-1}$} &
        \multicolumn{3}{c|}{$10^{-4}$}\\
     \hline   
     % \hline
     \multicolumn{1}{|c||}{\emph{\textbf{MAP torque estimation}}} &
     \multicolumn{3}{c|}{$\bm \tau_{(model + \unit{6}{\kilo}{\gram})}$} &
       \multicolumn{3}{c|}{$\bm \tau_{model}$}\\
      \hline
    \end{tabular}
\end{table}
%
By exploiting the linearity property of the system we started by considering the 
following expression for the torques
\begin{eqnarray} \label{tauEq}
	\bm \tau_{(model + \unit{6}{\kilo}{\gram})} - \bm \tau_{model} = 
	\bm \tau_{\unit{6}{\kilo}{\gram}}
\end{eqnarray} 
where $\bm \tau_{\unit{6}{\kilo}{\gram}}$ is the theoretical torque due to the additional $W$
positioned on the torso\footnote{We consider a simple 2-DoF system (see \cite{LatellaSensors2016}) in which the position of $W$ and the hip joint angle are known.}.  Given \eqref{tauEq}, 
 it is possible to retrieve the error $\bm{\varepsilon}_{\bm{\tau}}$ on the $\bm \tau$ 
  estimation for the subject with $W$, due to \emph{case A}:
  %
\begin{eqnarray} \label{TorquesError}
	\bm{\varepsilon}_{\bm{\tau}} = |\bm \tau_{(model + \unit{6}{\kilo}{\gram})} -
	 \bm \tau_{model}| - \bm \tau_{\unit{6}{\kilo}{\gram}}
\end{eqnarray}
%
We computed \eqref{TorquesError} by using the OpenSim ID (Inverse Dynamics) toolbox 
as well, in order to evaluate its effectiveness with respect to the modeling errors.
Figures \ref{fig:MAPtau_cmp}-\ref{fig:OPENSIMtau_cmp} provide the mean and the standard deviation of the torque estimation by means of the MAP algorithm and the OpenSim software, respectively.  Figure \ref{fig:BOXPLOT} shows the comparison between the computation of the error in \eqref{TorquesError} for both the above methods: the error is higher in OpenSim computation than in MAP since OpenSim does not offer the possibility of setting the model reliability in the computation.
%
% \begin{figure}[h!]
%   \centering
%    \begin{subfigure}[b]{1\columnwidth}
%       \includegraphics[width=\textwidth]{figs/TorqueComparison.pdf}
%           \caption{}
%           \label{fig:figs_torqueRightHipBH}
%   \end{subfigure}
%    \begin{subfigure}[b]{1\columnwidth}
%     \includegraphics[width=\textwidth]{figs/TorqueComparisonOPENSIM.pdf}
% 	\caption{}
%         \label{fig:figs_torqueRightKneeSR}
%    \end{subfigure}
%           \caption{\emph{Intra-subjects analysis}: }
% \end{figure}
%
 \begin{figure*}[!ht]
	 \centering
%	\begin{subfigure}[b]{0.33\textwidth}
		\includegraphics[width=0.3\textwidth]{figs/TorqueComparison}
%		\caption{}
%		\label{fig:MAPtau_cmp}
%	 \end{subfigure}
% 	\begin{subfigure}[b]{0.33\textwidth}
 		\includegraphics[width=0.3\textwidth]{figs/TorqueComparisonOPENSIM.pdf}
% 		\caption{}
%		\label{fig:OPENSIMtau_cmp}
% 	 \end{subfigure}
%  	\begin{subfigure}[b]{}
  		\includegraphics[width=0.3\textwidth]{figs/OPENSIMvsMAP}
%  		\caption{}
%		\label{fig:BOXPLOT}
%  	 \end{subfigure}
     \caption{\emph{Intra-subjects analysis}: mean and standard deviation of $\bm \tau$
	  (i.e., the sum of the $\bm \tau$ estimated at the hips)
	 among five repetitions of the $BT$ performed by a subject computed for \emph{case A} 
	 (red) and \emph{case B} (blue) by means of (a) MAP algorithm  and (b) by using the 
	 OpenSim ID (Inverse Dynamics) toolbox. (c) Box plots of the torque estimation error
	 $\bm{\varepsilon}_{\bm{\tau}}$ computed in \eqref{TorquesError} with MAP (on the right)
	 and with OpenSim (on the left). It shows that MAP is a method more robust to the
	 modelling errors since it gaves the possibility of weighting the reliability of the model
	  by properly setting the related covariance matrix. }
 \end{figure*}

%%%%%%%%%%%%%%%%%%%%%%%%%%%%%%%%%%%%%%%%%%%%%%%%%%%%%%%%%%%%%%%%%%%%%%%%%%%%%%%%%%%%%%%%%%%%%%%%
\subsection{Incremental sensor fusion analysis}
Since a distinctive feature of our framework consists in the possibility of building
 \eqref{eq:measRNEA} for different sources of measurement, we here investigate the advantage in
  using this algorithm for the dynamics estimation. We consider three sets of $\bm y$ 
  equations for: 
\begin{enumerate}
	\item  the two force plates: 
	\begin{eqnarray}
		\bm{y}_{\mbox{\scriptsize{\emph{FP}}}} & = & {}^{{\mbox{\scriptsize{\emph{FP}}}}}
		\bm{X}_{\mbox{\scriptsize{\emph{hFOOT}}}}~\bm{f}_{\mbox{\scriptsize{\emph{hFOOT}}}}
	\end{eqnarray}
	where it is used the trasformation matrix\footnote{See \cite{Featherstone2008}
	 for the definition of the trasformation
	 matrix between two reference frames.} $\bm X$  between the human foot reference frame 
	(${\mbox{\footnotesize{\emph{hFOOT}}}}$) and each force plate frame
	 (${\mbox{\footnotesize{\emph{FP}}}}$);
	\item  the IMUs embedded in the suit:
	\begin{eqnarray}
		\bm{y}_{\mbox{\scriptsize{\emph{IMU}}}} & = &
		 {}^{\mbox{\scriptsize{\emph{IMU}}}}\bm{X}_{\mbox{\scriptsize{\emph{L}}}}~
		 \bm{a}_{\mbox{\scriptsize{\emph{L}}}}
	\end{eqnarray}
	by exploiting the trasformation between each human link frame
	 ${\mbox{\footnotesize{\emph{L}}}}$ 
	on which the IMU is attached and
	that particular IMU reference frame (Fig. \ref{fig:figs_human_JointLink}a);  
	\item  the force/torque sensors of the two arms of the robot:
	\begin{eqnarray}
		\bm{y}_{\mbox{\scriptsize{\emph{iCubFT}}}} & = &
		 {}^{\mbox{\scriptsize{\emph{iCubFT}}}}\bm{X}_{\mbox{\scriptsize{\emph{hHAND}}}}~
		 \bm{f}_{\mbox{\scriptsize{\emph{hHAND}}}}
	\end{eqnarray}
	for which it is necessary knowing the transformation between each human hand frame
	(${\mbox{\footnotesize{\emph{hHAND}}}}$)
	to the robot sensor frame (${\mbox{\footnotesize{\emph{iCubFT}}}}$).
\end{enumerate}

\indent
A general overview of the above-mentioned frames is shown in Fig.
 \ref{fig:interaction_lateral&top}a. We want to prove that, by adding progressively 
 the different sensors data at each MAP computation, the variance associated to 
 the estimated dynamic variables consequently decreases, making the estimation more reliable.
%
In particular, we build \eqref{eq:measRNEA} for three different cases (Fig.
 \ref{fig:sensAddition&barAnalysis}a) :
\begin{itemize}
\item[\textit{case 1)}]  $\bm{y} = [\bm{\ddot{q}},~\bm{y}_{\mbox{\scriptsize{\emph{FP}}}}]$
\item[\textit{case 2)}] $\bm{y} = [\bm{\ddot{q}},~\bm{y}_{\mbox{\scriptsize{\emph{FP}}}},~ \bm{y}_{\mbox{\scriptsize{\emph{IMUs}}}}]$ 
\item[\textit{case 3)}] $\bm{y} = [\bm{\ddot{q}},~\bm{y}_{\mbox{\scriptsize{\emph{FP}}}},~ \bm{y}_{\mbox{\scriptsize{\emph{IMUs}}}},~\bm{y}_{\mbox{\scriptsize{\emph{iCubFT}}}}]$.
\end{itemize}
%
The MAP computation is performed for each case since the incremental addition of a sensor 
includes each time a new information on the analysis.  
For this analysis we take into account a task involving the robot ($BT$) in order to 
include the robot sensor measurements in the computation.  Among the variables in $\bm d$ we consider again the torque $\bm \tau$ (i.e., right and left ankle and hip) and the variance along the axis of major relevance $y$.

 \begin{figure*}[!ht]
	 \centering
		\includegraphics[width=0.9\textwidth]{figs/sensAdditionAndBarAnalysis}
      \caption{(a) Description of three cases for progressive addition of sensors.
	   (b) \emph{Inter-subjects analysis}: mean variance of $\bm \tau$ at the left and right
	    ankle and hip among five repetitions of the $BT$ performed by ten subjects computed 
		by MAP with the three different version of the measurements vector $\bm y$.}
		 		\label{fig:sensAddition&barAnalysis}
 \end{figure*}
% 
 % \begin{figure*}[h!]
 % 	 \centering
 % 	\begin{subfigure}[b]{0.48\textwidth}
 % 		\includegraphics[width=\textwidth]{figs/sensAddition.pdf}
 % 		\caption{}
 % 		\label{fig:figs_sensAdd}
 % 	 \end{subfigure}
 % 	\begin{subfigure}[b]{0.48\textwidth}
 % 		\includegraphics[width=\textwidth]{figs/SDall.pdf}
 % 		\caption{}
 % 		\label{fig:figs_SDall}
 % 	 \end{subfigure}
 %      \caption{(a) Description of three cases for progressive addition of sensors.
 % 	   (b) \emph{Inter-subjects analysis}: mean variance of $\bm \tau$ at the left and right
 % 	    ankle and hip among five repetitions of the $BT$ performed by ten subjects computed
 % 		by MAP with the three different version of the measurement vector $\bm y$.}
 % \end{figure*}
 %
Passing progressively from \textit{case 1} to \textit{case 3} (Fig. \ref{fig:sensAddition&barAnalysis}a)
the variance associated to the torques decreases\footnote{In order to assess 
the statistical significance of results, a paired-samples
 \emph{t-test} is performed firstly between \textit{case 1} and \textit{case 2} 
 ($2$ sensors vs $3$ sensors) and then between \textit{case 2} and \textit{case 3} 
 ($3$ sensors vs all sensors).  Torque variances statistically significant,  
  \emph{p-value} $<0.05$.}. In Fig. \ref{fig:sensAddition&barAnalysis}b we show 
the decreasing behaviour of the mean variance of the torque at the hips and at the ankles
 computed between ten subjects.
%
The variance values on the ankles do not change significantly among the three different
 configurations of sensors since the ankle torque estimation depends mostly on the 
 contribution of the force plates that are included in all the three 
 cases of the computation.  Conversely, a significant decreasing behaviour is present 
 in the values associated to the hips. In this case the contribution of the three sources 
 of sensors becomes important since the torque estimation at the hips are affected 
 by the all sensors.
%!TEX root = ../template.tex

\section{Conclusions and Future Works}
%
One of the aim of the presented framework is the estimation of dynamic variables of 
a human being while is physically interacting with a robot.
 This paper is the extended version of the probabilistic framework explored in
  \cite{LatellaSensors2016} but we introduced here a more complex model ($23$ vs $3$
   links and $22$ vs $2$ joints) and we found that our framework is able to estimate
    human dynamics variables even in presence of a multi-DoFs model.  We performed
	 also a comparison with the well-established biomechanical software OpenSim:
	  the results were promising depicting a good bent of our framework in
 modelling possible inaccuracies in the model.
\\
\indent
In this paper the endeavour was to retrieve an estimation of the human dynamics by means of
 the MAP algorithm and to this purpose the robot was considered as a \emph{passive} forces
  measurer.  But since the human dynamics is of pivotal importance
   for a control design aimed at considering the \emph{human in the loop}, the forthcoming
    idea will be to provide in real-time the robot with the human force feedback
that could be used either as a tool for \emph{reactive} human-robot collaboration
(implying a robot reactive control) and, in a long-term perspective,
for \emph{predictive} collaboration, for enhancing remarkably the interaction naturalness.
Thus, in the near future a new robot controller has to be 
designed in order that the robot can adapt and adjust the interaction strategy accordingly.
 \\
 \indent
 In this work we applied the proposed approach to human models composed of 1-DoF revolute joints, 
 by using the classical formalism widespread in robotics \cite{Featherstone2008}.
 In particular we combined this type of joints to obtain a series of joints with a 
 high number of DoFs, that however are only a rough approximation of the complexity 
 exhibited by real-life biomechanical joints.
 While we chose this joint model for an initial exploration of the method, the proposed
  algorithm is not limited to this particular choice.
 In particular the properties of any joint (with an arbitrary number of DoFs) can be 
 encapsulated in an interface where the relative position, velocity and acceleration of 
 the two bodies connected by it are described by arbitrary functions of joint coordinates
  variables and their derivatives that can then be directly inserted in
   \eqref{eq:matRNEA} and \eqref{eq:measRNEA}. 
 In this way, any kind of joint modelling can be described under this formalism 
 (e.g., complex non-linear models, models of biomechanical joints based on tabulated data 
 obtained from human experiments \cite{delp1990interactive}). 
 A possible implementation is presented in
  \cite{seth2010minimal}, where it is used in the context of biomechanical simulations. 
 In the future we plan to translate this type of results in our framework, 
 to generalize our method to arbitrarily complex musculoskeletal models.
 
 
 
 
 
 
 
 
 
 
  
  


\bibliographystyle{abbrv}
\bibliography{D5.4}
\end{document}

%%% Local Variables:
%%% mode: latex
%%% TeX-master: t
%%% save-place: t
%%% End:
