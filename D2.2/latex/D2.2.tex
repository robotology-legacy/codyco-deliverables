%% Template for EU deliverable, using the deliverable.sty style file

\documentclass[12pt,a4paper,twoside]{report}

%% common package
\usepackage[headers]{deliverable}
\usepackage{xspace}
\usepackage{verbatim}
\usepackage[usenames]{color}
\usepackage[usenames,dvipsnames]{xcolor}
%\usepackage{graphicx}
\usepackage{url}
\usepackage{array}
\usepackage{graphics} 		% for pdf, bitmapped graphics files
\usepackage{graphicx}
\usepackage{epsfig} 		% for postscript graphics files
\usepackage{epstopdf}
\usepackage{caption}
\usepackage[labelformat=simple]{subcaption}
\renewcommand\thesubfigure{(\alph{subfigure})}
\usepackage{nameref}
\usepackage{multirow}
\usepackage{color}
%\usepackage{subfigure}
%%

%%insert here other packages needed by sections
%\usepackage{esint}
\usepackage{amstext}
\usepackage{amsmath}	 	% assumes amsmath package installed
\usepackage{amssymb}  		% assumes amsmath package installed
%\renewcommand\thesubfigure{(\alph{subfigure})}
\usepackage{wrapfig}
\usepackage[]{nomencl}		% nomenclatures
%\usepackage{dsfont}
%%

%%%%%%%%%%%%%%%%%%%%%%%%%%%%%%%%%%%%%%%%%%%%%%%%%%%%%%%%%%%%%%%%%%%%%%%%%%%%%%
%%% Special Commands and Packages needed by Elmar
%%%%%%%%%%%%%%%%%%%%%%%%%%%%%%%%%%%%%%%%%%%%%%%%%%%%%%%%%%%%%%%%%%%%%%%%%%%%%%
\usepackage{bm}
\usepackage{amsfonts}
\usepackage{setspace}
\usepackage{mathtools}
\newcommand{\FigureAbbr}{Fig~}%PLOS CB style
\newcommand{\FigureAbbrP}{Fig~}%PLOS CB style
\newcommand{\N}{\ensuremath{\mathcal{N}}}
\renewcommand{\vec}[1]{\ensuremath{\boldsymbol{#1}}}

%%%%%%%%%%%%%%%%%%%%%%%%%%%%%%%%%%%%%%%%%%%%%%%%%%%%%%%%%%%%%%%%%%%%%%%%%%%%%%
%%% Titlepage
%%%%%%%%%%%%%%%%%%%%%%%%%%%%%%%%%%%%%%%%%%%%%%%%%%%%%%%%%%%%%%%%%%%%%%%%%%%%%%

% declaration of variables used in style
\deliverableDocnumber{D2.2}
\deliverableTitle{Models of human whole body motions in contact with rigid and
  compliant support surfaces}

\deliverableAuthor{Morteza Azad$^{1}$, Michael Mistry$^{1}$ and Jan
  Babi\v{c}$^{2}$}
\deliverableResponsiblePartner{JSI}
\deliverableAffiliation{$^1$ School of Computer Science, University of
  Birmingham, UK.
  $^2$ Department for Automation, Biocybernetics and Robotics, Jo\v{z}ef
  Stefan Institute, Ljubljana, Slovenia.}

\deliverableReviewer{}
\deliverableCoordinator{Jan Babi\v{c}}
\deliverableActivityNumber{n} %% n=1,..,10
\deliverableActivity{Activity Name}
\deliverableDoctype{Deliverable} %% or Prototype
\deliverableClassification{Public} % or Consortium
\deliverableDistribution{Consortium} %
\deliverableStatus{Final} % Draft or Final
\deliverableDeliveryDate{29/2/2016}
\deliverableFile{Dxx\_DeliverableName.pdf} % please do not use "-" in the name
\deliverableVersion{1.0}
\deliverableDate{29/2/2016}
\deliverableYear{2016}
\deliverablePages{\pageref{LastPage}}
\deliverableChangelog{v.0.1 & Feb 04, 2016 & Initial draft %%\\\hline
%%              v.2.0 & Feb 20, 2007 & Final version
}
\deliverableProjectStartingDate{1st March 2013}
\deliverableProjectEndDate{28th February 2017}
\deliverableProjectAcronym{CoDyCo}
\deliverableProjectTitle{Whole-Body Compliant Dynamical Contacts in Cognitive Humanoids}
\deliverableContractNumber{600716}
\deliverableProjectCoordinator{Istituto Italiano di Tecnologia}
\deliverableProjectUrl{www.codyco.eu}
\deliverableFrameworkProgramme{FP7}

\deliverableWorkpackage{WP2}
\deliverableEditors{Morteza Azad, Michael Mistry and Jan Babi\v{c}}
\deliverableContributors{Morteza Azad (UB), Michael Mistry (UB), Chie
  Takahashi (UB), Elmar Rueckert (TUD), Jan Peters (TUD), Jernej \v{C}amernik
  (JSI), Jan Babi\v{c} (JSI)}
\deliverableReviewers{}
\deliverableAbstract{The scope of the current deliverable is to present the
  results on developing models of human whole body motions in contact with
  environment.}
\deliverableKeywordList{contacts, model learning, probabilistic movement
  representations}


\def\BibPath{./manuscript}       % location of bibtex-files
%%%%%%%%%%%%%%%%%%%%%%%%%%%%%%%%%%%%%%%%%%%%%%%%%%%%%%%%%%%%%%%%%%%%%%%%%%%%%%
%%% Sections
%%%%%%%%%%%%%%%%%%%%%%%%%%%%%%%%%%%%%%%%%%%%%%%%%%%%%%%%%%%%%%%%%%%%%%%%%%%%%%

%% constants
\newcommand{\botegoCaps}{BOTEGO}
\newcommand{\certhCaps}{CERTH}
\newcommand{\cybionCaps}{CYBION}
\newcommand{\nuigCaps}{NUIG}
\newcommand{\ubitechCaps}{UBITECH}

%%
%%%%%%%%%%%%%%%%%%%%%%%%%%%%%% BEGIN DOCUMENT
\begin{document}

\deliverableMaketitle

%%TODO move to style
\newcolumntype{L}[1]{>{\raggedright\let\newline\\\arraybackslash\hspace{0pt}}m{#1}}
\newcolumntype{C}[1]{>{\centering\let\newline\\\arraybackslash\hspace{0pt}}m{#1}}
\newcolumntype{R}[1]{>{\raggedleft\let\newline\\\arraybackslash\hspace{0pt}}m{#1}}

\textbf{Document Revision History}
\begin{center}
\begin{tabular}{|C{2cm}|C{3cm}|p{5cm}|C{4cm}|}
\hline
\textbf{Version}&\textbf{Date}&\textbf{Description}&\textbf{Author}\\\hline
v. 1.0 & 29 Feb 2016 & Final Version & Morteza Azad\\\hline
\end{tabular}
\end{center}
 
 \clearpage

\newpage
\renewcommand*\contentsname{Table of Contents}
\renewcommand*\listfigurename{Index of Figures}
\tableofcontents
\newpage
%\listoffigures
%\newpage

%%%%%%%%%%%%%%%%%%%%%%%% Start deliverable content here.

\chapter{Introduction}

%Short summary of our collaborative contribution

This deliverable summarizes the contribution of the CoDyCo consortium in tasks
T2.2 and T2.3 at the end of the third year.  These tasks are \textit{design of
  models for human whole body motion in contact}, and \textit{strategies of
  dealing with uncertainties in contact}, respectively.  The results are
briefly explained in four chapters.  In chapter 2, dynamic manipulability of
the centre of mass (CoM) is introduced as a metric for measuring the balance
ability of legged robots while they are in contact with their environment.
Experiments on human subjects show the applicability of this theory in
analysing balancing recovery motion in humans.  In chapter 3, the effects of
using handles is studied for posture control of standing subjects while they
are perturbed by external forces.  It is shown that the use of handles
significantly reduces the displacement of the centre of pressure.  Also, it is
observed that subjects clearly relied on using the handle for support, even
though the perturbations did not pose a significant balance threat.  Chapter 4
investigates the use of supporting contact (and its relationship with task)
while it is required for accomplishing a task.  In the experiments, human
subjects are asked to reach a target by the motion of one arm while they has
to maintain their postural balance with the other arm.  In a novel movement
model, strong correlations between both arms are found which were used to
predict the reaching motion solely from observing the motion related to
keeping postural stability. This finding has the potential to impact pre-tests
of central nervous system disorders that are less prone to factors like
stress, sleep deprivation and age compared to the classical cognitive tests.
In robotics, the model can be exploited to overcome current limitations of
autonomous robots in interacting with the environment through supportive
contacts.

% \chapter{Executive Summary}

\chapter{Manipulability of the Center of Mass: A Tool to Study, Analyse and
  Measure the Ability to Balance (UB/JSI)}\label{sec:Morteza}

This chapter introduces a set of metrics to study, analyse and measure the
ability to balance for both humans and legged robots.  This set of metrics,
which we call manipulability of the center of mass, are defined based on the
concept of end-effector manipulability in the literature.  Regarding the
center of mass as an end-effector and using impulsive dynamics, the metrics
are calculated to study the ability to move and accelerate this point.  They
graphically show the instantaneous change of the center of mass velocity due
to the unit weighted norm of instantaneous changes of the joint velocities or
impulses at the joints.  The proposed metrics can be computed for humans and
general legged robots with floating base and multiple contacts with the
environment in 3D space.  This chapter also provides the results of
experiments on humans to verify the application of the metrics.  In the
experiments, the centers of mass of human subjects in different configurations
are perturbed and joint torques are computed by using inverse dynamics.  The
metrics are shown to be suitable for comparing different postures in the sense
of the total required effort for balance maintenance.

Further details are in the confidential appendix.






\chapter{Postural Control Precedes and Predicts Volitional Motor Control
  (TUD/JSI)}\label{sec:ElmarPrePrint}


Supportive hand contacts are essential for mastering every-day life tasks, for 
example, when reaching for a glass on the highest shelf humans typically have to 
use the other hand to support their body on the kitchen table. In such 
scenarios, the motion of the body and both arms have to be perfectly 
synchronized to successfully perform the reaching motion and to simultaneously 
ensure the postural stability. However, little is known about the underlying 
processes that govern the motion of the human body during and after the learning 
of these kinds of concurrent motor skills. To study the effect of supportive 
contacts on motor control of reaching, an innovative full-body experimental 
paradigm was established that extends current experimental methods to a more 
ecological setting. The task of the subjects was to reach with their right arm 
for a distant target on a screen while postural stability could only be 
maintained by establishing an additional supportive hand contact with their left 
arm. To examine adaptation, non-trivial postural perturbations of the subjects' 
support base were systematically introduced. A novel probabilistic trajectory 
model approach was employed to analyze the correlation between the motions of 
both arms. We found that subjects adapted to the perturbations by establishing 
supportive hand contacts that were dependent on the location of the reaching 
target. Moreover we found that the trunk motion adapted significantly faster 
than the motion of the arms. However, the most striking finding was that 
observations of the initial phase of the left arm or trunk motion (100-400 ms) 
were sufficient to faithfully predict the complete movement of the right arm. 
Overall, our results suggest that the goal-directed arm movements determine the 
supportive arm motions that ensure postural stability and that adaptation 
happens on different time scales, where the motion of heavy body parts adapts 
faster than light arms. 

The details are in the confidential appendix.



\chapter{Use of rigid contacts during continuous perturbations
  (JSI)}\label{sec:Jernej}
\renewcommand{\thesection}{\arabic{section}}  
\renewcommand{\thetable}{\arabic{table}}  
\renewcommand{\thefigure}{\arabic{figure}} 
\renewcommand{\theequation}{\arabic{equation}} 


Standing balance in human everyday environment is often exposed to
unpredictable and continuous external perturbations.  Moreover, when postural
control is impaired or challenged, handrails, canes, and handles are often
used to assist maintaining balance and the effects of these firm supportive
contacts in such conditions should be considered.  Therefore, we examined
changes in postural control in response to continuous, unpredictable
perturbations and explored the effect of using a handle as a supportive
contact.  Postural control of standing subjects was assessed with measurements
of centre of pressure (COP), which we also compared with perturbation waveform
and forces exerted on the handle, to check for correlations.  Kinematic data
were used to determine changes in posture and electromyographic data to define
the magnitude of muscle activity.  The use of handle affected the control of
posture by reducing the excursions of COP.  The reduction was found to be more
reflective in the posterior direction of COP excursions and was also in line
with higher forces exerted on the handle in the same direction.  The change of
posture was immediate when the contact to the handle was omitted and
significantly different between the two conditions.  Muscle activation levels
of the trunk flexor were significantly higher in the hand supported trial.  In
summary, we found that subjects clearly relied on using the handle for
support, even though the perturbations did not pose a significant balance
threat.  Results of direction specific control of posture with hand support
can be considered in rehabilitation and fall prevention programmes.


\section{Introduction}
Postural control is one of the vastly investigated area of human motor control in the last few decades. Most of the research on postural control focused on the role of sensory input in maintaining postural control during quiet standing, and in response to external balance perturbations. However, in our daily lives handrails, canes and handles are often used to assist maintaining balance, since they provide additional supportive contacts with the environment.
With respect to the use of hand contacts for postural control, one of the most investigated phenomena is "light touch" \cite{Jeka1997,Krishnamoorthy2002}. These light, fingertip contacts with stationary objects provide an additional sensory input, which helps individuals to better position them in space \cite{Jeka1997}. Furthermore, a more accurate sensory information improves postural control by reducing the amplitude of the centre of pressure (COP) movement \cite{Jeka1997,johannsen2007effects,Kouzaki2008,Wing2011a}.
However, hand contacts can serve as more than just sensory input. In situations where balance is exposed to larger perturbations, such as experienced on a moving bus or train, a firm hand contact (i.e. holding) is needed, as it provides a much better stabilising potential than a light touch \cite{Maki1997}. Holding to a handle, besides increasing the base of support of a standing individual, also enables generation of forces at the hand to counteract such perturbations \cite{Sarraf2014,Babic2014b}. For this, the location of the handle with respect to the subject’s position is important. Babič et al. \cite{Babic2014b} recently found that handle position relative to the subject, along with support surface perturbation direction and intensity, has a significant effect on the maximal forces exerted at the handle during support surface perturbations in quiet standing. More specifically, lower forces exerted at handles located at shoulder and eye level were needed to maintain a comparable peak displacement of the COP. This indicates that handles were used for postural control irrespective of their position, but certain handle positions could be exploited more efficiently.
Previously mentioned studies all based on discrete perturbations of balance. Such perturbations evoke reactive postural responses and conclusions were made on the basis of these responses. However, a major component of such responses is comprised of motor actions that are related to various sensorimotor reflexes and in less extent to the voluntary component of the postural control [13].
In contrast to discrete perturbations, continuous perturbations involve both reactive and proactive components of motor actions and in this sense offer a complementary insight into the postural control. Therefore, in this research we focused on changes in postural control during continuous perturbations of subject's balance. We aimed to record the mechanical function of the arm in a function of whole body balance stabilization to measure the effect of a firm hand contact on postural control.


\section{Methods}
We measured thirteen healthy right-handed young adults (average age = 22.2 years, SD = 2.2 years, average height 179 cm, SD = 6.2 cm and average weight = 76.7 kg, SD = 8.4 kg). The study was previously approved by the National medical ethics committee (No. 112/06/13) and all subjects participated after giving their written consent. Data of three subjects were  excluded from the analyses due to some technical issues.

\subsection{Experimental protocol}
Subjects were standing on a force plate while their standing balance was being perturbed for 5 minutes by a motorized waist-pull system \cite{Peternel2013} (\FigureAbbr \ref{fig:methods}). They were required to keep upright with their feet placed at hip width, look straight ahead, and maintain balance without making any unnecessary corrective steps. The experiment consisted of two conditions: balancing with ("with handle"; WH) and without ("no handle"; NH) holding onto a handle. In the WH condition subject held onto a stationary handle (diameter $=$ 3.2 cm, length $=$ 12 cm) positioned at shoulder height \cite{Babic2014b} with their right hand. In the NH condition subjects were standing freely with their arms folded across their chest. 
Balance was perturbed using a random white noise signal constructed to emulate mild, daily life perturbations (e.g., public transport) bus and avoid large, abrupt, and startling balance perturbations. This perturbation signal was based on pilot experiments, had a frequency range between 0.25 and 1.00 Hz and the maximum perturbation force of 11\% of the subject's body weight. 
Kinetic data were collected using a force plate (9281CA, Kistler Instrumente AG, Winterthur, Switzerland) under the subjects’ feet and a 3-axis force sensor (45E15A, JR3, Woodland, USA) on the base of the handle, both at 1000 samples/s. Unilateral  (right hand side) kinematic data were collected at a sampling rate of 100 samples/s using a contactless motion capture system (3D Investigator, Northern Digital Inc., Waterloo, Ont., Canada) consisting of a 3$\times$3 camera array. Seven active markers were attached on the subject's right 5th metatarsal-phalangeal, ankle, knee, hip, shoulder, elbow and wrist joint. Electromyographical (EMG) activity of the right leg (Tibialis Anterior; TA, Gastrocnemius lateralis: GA) and of the trunk (Multifidus; MF, Obliques Externus; OE) was measured using Biometrics DataLOG MW8X at a sampling rate of 1000 samples/s. 
Before the start of the experiment, subjects performed three maximal voluntary contractions (MVC) of each of the measured muscles and were exposed to 14 trials of 5 minutes of the WH condition. The aim of these preparatory WH trials was to familiarize the subjects with the experimental set-up and avoid any learning effects observed in previous balance experiments.

\begin{figure}[htb!]
	\centering
	\includegraphics[width=0.85\textwidth]{Jernej/figures/expSetup}
	\caption{\textbf{Experimental setup. }The subject is standing on a force plate, wearing a waist belt connected to the motorized waist-pull system which generated translational force perturbations in the anterior-posterior direction using a random white noise signal constructed to emulate mild, daily life perturbations. The actual perturbation waveform is shown on the plot below the motorized waist-pull system.
			}
	\label{fig:methods}
\end{figure}

\subsection{Data analysis}  
Anteroposterior  displacement of the subject’s centre of pressure (COP) was calculated from the data provided by the force plate on which the subjects were standing. 
Kinematic data were low pass filtered (zero lag, 2nd order Butterworth algorithm, cut-off frequency 20 Hz) \cite{Bartlett1997} and ankle, knee, and hip joint angles were calculated from the joint markers coordinates. 
Mean values of joint angles over time were fitted using an exponential model $y = A^{e-t/\tau} + C$, where $A$ is the gain of the exponential process, $\tau$ is the time constant, $C$ is the offset, and $t$ refers to the trial number) to describe evaluation  of motor adaptation over time \cite{Franklin2003}. The onset of reaching a plateau (adaptation stabilized) was defined by calculating point in time at the three time constants (3$\tau$) of the fitted exponential curve. This is the point when the function reaches a value of less than 5\% of its starting value and was considered as the adaptation stabilized. 
EMG was band-pass filtered (zero lag, 2nd order Butterworth algorithm, with cut-off frequencies of 20 and 450 Hz), full-wave rectified and normalized by division with the MVCs.  By applying a low pass filter (zero lag, 2nd order Butterworth algorithm, 10 Hz cut-off frequency), we created envelopes of EMG signals and then integrated them over time, to observe the accumulated EMG activity.
\subsection{Statistical analysis}
To compare the NH and WH conditions we calculated the average COP displacement, hip, knee, and ankle angles and contact forces exerted on the handle over the 5 minutes for each subject. These individual average values were used for statistical analysis. 
We used paired samples t-test analysis to investigate the differences between the WH and NH conditions and linear correlation to investigate the relationship between the COP displacement and the magnitude of the perturbation (separately for anterior and posterior directions) and between COP and the exerted handle contact force. All statistical analyses were performed using SPSS 21 Inc., Chicago, USA and statistical significance was set at $\alpha =$ 0.05. The effect size (\textit{d}) was calculated by using standard Cohen’s equation ($\hat{d}=\dfrac{\bar{X}_{1} - \bar{X}_{2}}{s}$) \cite{Cohen1988}. 

After comparing the two trials we checked for direction specific differences of COP displacement within each trial. For this, we used a paired samples t-test on averaged measures of anteroposterior COP displacement. The relationship between the perturbation and the COP displacement was investigated in more detail by correlating the perturbation force and COP displacement and by correlating perturbation force and forces on the handle.

\section{Results}
Average anteroposterior displacements of the COP during the NH and WH conditions are shown in \FigureAbbr \ref{fig:cop}. In both conditions COP displacement was larger in the anterior direction (mean $\pm$ SE: NH 38.45 $\pm$ 1.6 mm,  WH 18.15 $\pm$ 1.2 mm) compared to posterior (mean $\pm$ SE: NH $-$34.88 $\pm$ 2 mm, WH $-$11.02 $\pm$ 1.5 mm), but this difference was significant only for the WH condition (\textit{t}(9) $=$ 2.81, \textit{p} $=$ .02, \textit{d} $=$ 1.52). Hence, the remainder of our COP analyses were conducted for the anterior and posterior directions separately. 

Overall, COP displacements were significantly larger in the NH condition compared to WH condition, both in the anterior (difference of 20.3 mm, \textit{t}(9) = 7.78, \textit{p} = .001, \textit{d} = $-$4.15) and posterior direction (difference of 23.9 mm, \textit{t}(9) = $-$11.09, \textit{p} = .001, \textit{d} = $-$3.8). 

\clearpage
As can be seen from \FigureAbbr \ref{fig:corr}A, the correlation between the COP displacement and perturbation force was \textit{r}$_{p} =$ .77 (\textit{p} $<$ .001) and \textit{r}$_{a} =$ .82 (\textit{p} $<$ .001) in the NH condition for the posterior and anterior direction, respectively. For the WH condition (\FigureAbbr \ref{fig:corr}B) the correlations were \textit{r}$_{p} =$ .67 (\textit{p} $<$ .001) and \textit{r}$_{a} =$ .89 (\textit{p} $<$ .001) for the posterior and anterior direction, respectively.

\begin{figure}[!htb]
	\centering
	\includegraphics[width=0.7\textwidth]{Jernej/figures/cop}
	\caption{\textbf{Anteroposterior displacement of COP. }Mean COP displacement in NH and WH trials, for the anterior (positive) and posterior (negative) directions. Error bars indicate $\pm$ 1 standard error of the mean.
	}
	\label{fig:cop}
\end{figure}

Correlations between the forces exerted on the handle and the perturbation force (\FigureAbbr \ref{fig:corr}C) were large in both anterior (\textit{r}$_{p}$ $=$ .85, \textit{p} $<$ .001) and posterior direction (\textit{r}$_{a}$ $=$ .81, \textit{p} $<$ .001). However, the slope of a least-squares linear fit to the data indicates, that subjects utilized the handle considerably more for perturbations in the posterior direction (k$_{p}$ = 1.3) than for perturbations in the anterior direction (k$_{a}$ = .86).

\clearpage
\begin{figure}[!htb]
	\centering
	\includegraphics[width=0.7\textwidth]{Jernej/figures/corr}
	\caption{\textbf{Correlation. }\textbf{(A)} Correlation between the perturbation force and COP displacement in the NH condition, \textbf{(B)} correlation between COP displacement and perturbation force in the WH condition, and \textbf{(C)} correlation between handle force and perturbation force in the WH condition.  All correlations were calculated separately for the anterior (positive) and posterior (negative) direction.
	}
	\label{fig:corr}
\end{figure}

Joint angles prior to the start of perturbation were significantly smaller in the NH condition compared to WH condition (\FigureAbbr \ref{fig:jAnglesBars}). Differences were the largest in the knee (mean $\pm$ SE: 169.4 $\pm$ 1.4$^{\circ}$ for NH, 172.9 $\pm$ 1.4$^{\circ}$ for WH, \textit{t}(9) $= -$4.05, \textit{p} $=$ .01, \textit{d} $=$ 1.36 ), followed by the hip (mean $\pm$ SE: 179 $\pm$ 1.8$^{\circ}$ for NH, 181.6 $\pm$ 1.5$^{\circ}$ WH, \textit{t}(9) $= -$2.95, \textit{p} $=$ .024, \textit{d} $=$ 1.13), and ankle (mean $\pm$ SE: 110.4 $\pm$ 1.1$^{\circ}$ for NH, 112.1 $\pm$ 1.2$^{\circ}$ WH, \textit{t}(9) $= -$2.71, \textit{p} $=$ .038, \textit{d} $=$ 1.08).


\begin{figure}[!htb]
	\centering
	\includegraphics[width=0.7\textwidth]{Jernej/figures/jAnglesBars}
	\caption{\textbf{Ankle, knee and hip joint angles. }Mean value of joint angles during the perturbation is given for NH (blue bars) and WH condition (green bars) and mean joint angles prior to the start of perturbation are shown as red circles above bars. Error bars indicate $\pm$ 1 standard error of the mean.
	}
	\label{fig:jAnglesBars}
\end{figure}

Ankle (A), knee (B), and hip (C) angles over the time course of the perturbation are shown in \FigureAbbr \ref{fig:expFit}. Exponential curves fitted to the data show that mean joint angles in the NH condition changed after the perturbation onset before reaching a steady state. The steady state was reached first by the ankle angle (86 s after perturbation onset), followed by the knee angle (112 s after the perturbation onset) and finally hip angle (195 s after the perturbation onset), resulting in more ankle, knee and hip flexion.

\begin{figure}[!htb]
	\centering
	\includegraphics[width=0.92\textwidth]{Jernej/figures/expFit}
	\caption{\textbf{Ankle, knee and hip joint angles. }Figures represent mean ankle \textbf{(A)}, knee \textbf{(B)}, and hip \textbf{(C)} angles over the time course of the perturbation. Thin solid curves represent mean joint angles during NH and WH conditions. Thick solid lines represent exponential curve fit, denoting adaptation of joint angles in the NH (orange) and WH (blue) conditions, while shaded areas represent standard error of the mean. The dotted vertical lines represent perturbation onset (PO) while the dashed vertical lines indicate stabilized changes (adaptation) in the joint angles (SA).
	}
	\label{fig:expFit}
\end{figure}

\clearpage
Finally, muscle activity is significantly lower during the WH condition than during the NH condition both for the leg muscles (GA \textit{t}(9) $=$ 3.57, \textit{p} $=$ .04, \textit{d} $= -$0.89; TA \textit{t}(9) $=$ 6.41, \textit{p} $=$ .002, \textit{d} $= -$1.85) and trunk muscle (MF \textit{t}(9) $=$ 6.5, \textit{p} $=$ .001, \textit{d} $= -$1.01), as can be seen in \FigureAbbr \ref{fig:iemg}. 
Leg muscle activity is lower for 18.4 $\pm$ 4.9\% in the GA (mean $\pm$ SE: NH: 28.97 $\pm$ 6.5\% MVC, WH: 10.6 $\pm$ 2.3\% MVC) and for 23.7 $\pm$ 3.5\% in the TA (mean $\pm$ SE: NH: 27.21 $\pm$ 4\% MVC, WH: 3.47 $\pm$ 1.9\%), while trunk muscle activity is lower for 14.3 $\pm$ 2.1\% in the MF (mean $\pm$ SE: NH: 36.17 $\pm$ 4.5\% MVC, WH: 21.83 $\pm$ 3.7\%).

\begin{figure}[!htb]
	\centering
	\includegraphics[width=0.7\textwidth]{Jernej/figures/iemg}
	\caption{\textbf{Muscle activity. }Mean integrated EMG activity of leg (GA and TA) and trunk (OE and MF) muscles during the NH (green bars) and WH condition (blue bars). Error bars indicate $\pm$ 1 standard error of the mean.
	}
	\label{fig:iemg}
\end{figure}


\section{Conclusions}
Standing balance in human everyday environment is often exposed to unpredictable and continuous external perturbations. Moreover, when postural control is impaired or challenged, handrails, canes, and handles are often used to assist maintaining balance and the effects of these firm supportive contacts in such conditions should be considered.
Therefore, we examined changes in postural control in response to continuous, unpredictable perturbations and explored the effect of using a handle as a supportive contact. Postural control of standing subjects was assessed with measurements of centre of pressure, which we also compared with perturbation waveform and forces exerted on the handle, to check for correlations. Kinematic data were used to determine changes in posture and electromyographic data to define the magnitude of muscle activity.
COP displacement, hip, knee, and ankle angles, leg and trunk muscle activity and handle contact forces were analysed for the anterior and posterior directions separately, as COP displacement was significantly larger in the anterior direction (WH, 7 mm, \textit{p} $=$ .02). Perturbation force was strongly correlated to COP displacement (all r $>$ .65) and handle forces (r $>$ .8) in both directions. COP displacement was significantly larger in the NH condition compared to WH condition (anterior: 20 mm, posterior: 24 mm, both \textit{p} $=$ .001) and regression indicated that subjects utilized the handle slightly more for posterior perturbations. In the NH condition, all joint angles decreased in anticipation of the perturbation (2-4$^{\circ}$, all \textit{p} $<$ .04) and until 86-195 s following perturbation onset. Finally, leg (18-24\%) and one of the trunk (14\%) muscles increased their activity in the NH condition (all \textit{p} $\leqslant$ .04). 
In summary, we found that subjects clearly relied on using the handle for support, even though the perturbations did not pose a significant balance threat. Results of direction specific control of posture with hand support can be considered in rehabilitation and fall prevention programs.


\bibliographystyle{plain}
\bibliography{manuscriptAll,rigidContacts}




\end{document}

%%% Local Variables:
%%% mode: latex
%%% TeX-master: t
%%% save-place: t
%%% End:
