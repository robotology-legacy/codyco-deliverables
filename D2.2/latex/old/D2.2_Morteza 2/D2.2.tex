%% Template for EU deliverable, using the deliverable.sty style file

\documentclass[12pt,a4paper,twoside]{report}

%% common package
\usepackage[headers]{deliverable}
\usepackage{xspace}
\usepackage{verbatim}
\usepackage[usenames]{color}
\usepackage[usenames,dvipsnames]{xcolor}
%\usepackage{graphicx}
\usepackage{url}
\usepackage{array}
\usepackage{graphics} 		% for pdf, bitmapped graphics files
\usepackage{graphicx}
\usepackage{epsfig} 		% for postscript graphics files
\usepackage{epstopdf}
\usepackage{caption}
\usepackage[labelformat=simple]{subcaption}
\renewcommand\thesubfigure{(\alph{subfigure})}
\usepackage{nameref}
\usepackage{multirow}
\usepackage{color}
%\usepackage{subfigure}
%%

%%insert here other packages needed by sections
%\usepackage{esint}
\usepackage{amstext}
\usepackage{amsmath}	 	% assumes amsmath package installed
\usepackage{amssymb}  		% assumes amsmath package installed
%\renewcommand\thesubfigure{(\alph{subfigure})}
\usepackage{wrapfig}
\usepackage[]{nomencl}		% nomenclatures
%\usepackage{dsfont}
%%

%%%%%%%%%%%%%%%%%%%%%%%%%%%%%%%%%%%%%%%%%%%%%%%%%%%%%%%%%%%%%%%%%%%%%%%%%%%%%%
%%% Special Commands and Packages needed by Elmar
%%%%%%%%%%%%%%%%%%%%%%%%%%%%%%%%%%%%%%%%%%%%%%%%%%%%%%%%%%%%%%%%%%%%%%%%%%%%%%
\usepackage{bm}
\usepackage{amsfonts}
\usepackage{setspace}
\usepackage{mathtools}
\newcommand{\FigureAbbr}{Fig~}%PLOS CB style
\newcommand{\FigureAbbrP}{Fig~}%PLOS CB style
\newcommand{\N}{\ensuremath{\mathcal{N}}}
\renewcommand{\vec}[1]{\ensuremath{\boldsymbol{#1}}}


%%%%%%%%%%%%%%%%%%%%%%%%%%%%%%%%%%%%%%%%%%%%%%%%%%%%%%%%%%%%%%%%%%%%%%%%%%%%%%
%%% Titlepage
%%%%%%%%%%%%%%%%%%%%%%%%%%%%%%%%%%%%%%%%%%%%%%%%%%%%%%%%%%%%%%%%%%%%%%%%%%%%%%

% declaration of variables used in style
\deliverableDocnumber{D2.2}
\deliverableTitle{Models of human whole body motions in contact with rigid and compliant support surfaces}

\deliverableAuthor{Morteza Azad, Elmar Rueckert$^1$ and Jan Peters$^{1,2}$}
\deliverableResponsiblePartner{TUD}
\deliverableAffiliation{% Insert here authors affiliations
 $^1$ Intelligent Autonomous Systems Lab, Technische Universit\"at Darmstadt, 64289 Darmstadt, Germany.
  $^2$ Robot Learning Group, Max-Planck Institute for Intelligent Systems,
	Tuebingen, Germany.
}

\deliverableReviewer{}
\deliverableCoordinator{Jan Babi\v{c}}
\deliverableActivityNumber{n} %% n=1,..,10
\deliverableActivity{Activity Name}
\deliverableDoctype{Deliverable} %% or Prototype
\deliverableClassification{Public} % or Consortium
\deliverableDistribution{Consortium} %
\deliverableStatus{Draft} % Draft or Final
\deliverableDeliveryDate{29/2/2016}
\deliverableFile{Dxx\_DeliverableName.pdf} % please do not use "-" in the name
\deliverableVersion{1.0}
\deliverableDate{Feb.~29, 2016}
\deliverableYear{2016}
\deliverablePages{\pageref{LastPage}}
\deliverableChangelog{v.0.1 & Feb 04, 2016 & Initial draft %%\\\hline
%%              v.2.0 & Feb 20, 2007 & Final version
}
\deliverableProjectStartingDate{1st March 2013}
\deliverableProjectEndDate{28th February 2017}
\deliverableProjectAcronym{CoDyCo}
\deliverableProjectTitle{Whole-Body Compliant Dynamical Contacts in Cognitive Humanoids}
 \deliverableContractNumber{600716}
 \deliverableProjectCoordinator{Istituto Italiano di Tecnologia}
 \deliverableProjectUrl{www.codyco.eu}
 \deliverableFrameworkProgramme{FP7}
 
 \deliverableWorkpackage{WP2}
 \deliverableEditors{Morteza Azad and Jan Babi\v{c}}
 \deliverableContributors{Morteza Azad (UB) / Elmar Rueckert, Serena Invaldi, Jan Peters (TUDA) / Jernej Camernik, Jan Babic (JSI)}
 \deliverableReviewers{}
\deliverableAbstract{The scope of the current deliverable is to present the results on developing models of human whole body motions. 
}
\deliverableKeywordList{contacts, model learning, probabilistic movement representations}


\def\BibPath{./manuscript}       % location of bibtex-files
%%%%%%%%%%%%%%%%%%%%%%%%%%%%%%%%%%%%%%%%%%%%%%%%%%%%%%%%%%%%%%%%%%%%%%%%%%%%%%
%%% Sections
%%%%%%%%%%%%%%%%%%%%%%%%%%%%%%%%%%%%%%%%%%%%%%%%%%%%%%%%%%%%%%%%%%%%%%%%%%%%%%

%% constants
\newcommand{\botegoCaps}{BOTEGO}
\newcommand{\certhCaps}{CERTH}
\newcommand{\cybionCaps}{CYBION}
\newcommand{\nuigCaps}{NUIG}
\newcommand{\ubitechCaps}{UBITECH}

%%
%%%%%%%%%%%%%%%%%%%%%%%%%%%%%% BEGIN DOCUMENT
\begin{document}

\deliverableMaketitle

%%TODO move to style
\newcolumntype{L}[1]{>{\raggedright\let\newline\\\arraybackslash\hspace{0pt}}m{#1}}
\newcolumntype{C}[1]{>{\centering\let\newline\\\arraybackslash\hspace{0pt}}m{#1}}
\newcolumntype{R}[1]{>{\raggedleft\let\newline\\\arraybackslash\hspace{0pt}}m{#1}}

\textbf{Document Revision History}
\begin{center}
\begin{tabular}{|C{2cm}|C{3cm}|p{5cm}|C{4cm}|}
\hline
\textbf{Version}&\textbf{Date}&\textbf{Description}&\textbf{Author}\\\hline
v. 0.1 & Feb. 04 & Initial Draft & Morteza Azad\\\hline
v. 1.0 & Feb. 29 & Final & Morteza Azad\\\hline
\end{tabular}
\end{center}
 
 \clearpage

\newpage
\renewcommand*\contentsname{Table of Contents}
\renewcommand*\listfigurename{Index of Figures}
\tableofcontents
\newpage
%\listoffigures
%\newpage

%%%%%%%%%%%%%%%%%%%%%%%% Start deliverable content here.

\chapter{Introduction}

%Short summary of our collaborative contribution
\paragraph{Postural Control Precedes and Predicts Volitional Motor Control (TUD/JSI)} 

The human motor control system is involved in most learning tasks including 
movement skill acquisition, cognitive reasoning and all flavors of communication 
skills like speech, sign language, gestures and writing. It processes perceptual 
streams, recalls memorized movements and is modulated by anticipated goals. To 
minimize the effect of all these influences, most studies investigate motor learning 
in unnatural conditions where the results are difficult to exploit. We show that 
motor adaptation can be analyzed under more ecological and thus more natural 
conditions by inspecting  the adaptation of a full-body motor task. A target had 
to be reached by the motion of one arm and the postural stability had to be 
maintained by the other arm. In a novel movement model we found strong  
correlations between both arms which were used to predict the reaching motion 
solely from observing the motion related to keeping postural stability. This 
finding has the potential to impact pre-tests of central nervous system 
disorders that are less prone to factors like stress, sleep deprivation and age 
compared to the classical cognitive tests. In robotics, the model can be 
exploited to overcome current limitations of autonomous robots in interacting 
with the environment through supportive contacts. 

\chapter{Executive Summary}

\chapter{Postural Control Precedes and Predicts Volitional Motor Control (TUD/JSI)}\label{sec:ElmarPrePrint}


Supportive hand contacts are essential for mastering every-day life tasks, for 
example, when reaching for a glass on the highest shelf humans typically have to 
use the other hand to support their body on the kitchen table. In such 
scenarios, the motion of the body and both arms have to be perfectly 
synchronized to successfully perform the reaching motion and to simultaneously 
ensure the postural stability. However, little is known about the underlying 
processes that govern the motion of the human body during and after the learning 
of these kinds of concurrent motor skills. To study the effect of supportive 
contacts on motor control of reaching, an innovative full-body experimental 
paradigm was established that extends current experimental methods to a more 
ecological setting. The task of the subjects was to reach with their right arm 
for a distant target on a screen while postural stability could only be 
maintained by establishing an additional supportive hand contact with their left 
arm. To examine adaptation, non-trivial postural perturbations of the subjects' 
support base were systematically introduced. A novel probabilistic trajectory 
model approach was employed to analyze the correlation between the motions of 
both arms. We found that subjects adapted to the perturbations by establishing 
supportive hand contacts that were dependent on the location of the reaching 
target. Moreover we found that the trunk motion adapted significantly faster 
than the motion of the arms. However, the most striking finding was that 
observations of the initial phase of the left arm or trunk motion (100-400 ms) 
were sufficient to faithfully predict the complete movement of the right arm. 
Overall, our results suggest that the goal-directed arm movements determine the 
supportive arm motions that ensure postural stability and that adaptation 
happens on different time scales, where the motion of heavy body parts adapts 
faster than light arms. 

The details are in the confidential appendix.



\bibliographystyle{plain}
\bibliography{manuscriptAll}

\end{document}

%%% Local Variables:
%%% mode: latex
%%% TeX-master: t
%%% save-place: t
%%% End:
